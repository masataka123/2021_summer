\documentclass[dvipdfmx,a4paper,11pt]{article}
\usepackage[utf8]{inputenc}
%\usepackage[dvipdfmx]{hyperref} %リンクを有効にする
\usepackage{url} %同上
\usepackage{amsmath,amssymb} %もちろん
\usepackage{amsfonts,amsthm,mathtools} %もちろん
\usepackage{braket,physics} %あると便利なやつ
\usepackage{bm} %ラプラシアンで使った
\usepackage[top=30truemm,bottom=30truemm,left=25truemm,right=25truemm]{geometry} %余白設定
\usepackage{latexsym} %ごくたまに必要になる
\renewcommand{\kanjifamilydefault}{\gtdefault}
\usepackage{otf} %宗教上の理由でmin10が嫌いなので


\usepackage[all]{xy}
\usepackage{amsthm,amsmath,amssymb,comment}
\usepackage{amsmath}    % \UTF{00E6}\UTF{0095}°\UTF{00E5}\UTF{00AD}\UTF{00A6}\UTF{00E7}\UTF{0094}¨
\usepackage{amssymb}  
\usepackage{color}
\usepackage{amscd}
\usepackage{amsthm}  
\usepackage{wrapfig}
\usepackage{comment}	
\usepackage{graphicx}
\usepackage{setspace}
\setstretch{1.2}


\newcommand{\R}{\mathbb{R}}
\newcommand{\Z}{\mathbb{Z}}
\newcommand{\Q}{\mathbb{Q}} 
\newcommand{\N}{\mathbb{N}}
\newcommand{\C}{\mathbb{C}} 
\newcommand{\Sin}{\text{Sin}^{-1}} 
\newcommand{\Cos}{\text{Cos}^{-1}} 
\newcommand{\Tan}{\text{Tan}^{-1}} 
\newcommand{\invsin}{\text{Sin}^{-1}} 
\newcommand{\invcos}{\text{Cos}^{-1}} 
\newcommand{\invtan}{\text{Tan}^{-1}} 
\newcommand{\Area}{\text{Area}}
\newcommand{\vol}{\text{Vol}}




   %当然のようにやる.
\allowdisplaybreaks[4]
   %もちろん.
%\title{第1回. 多変数の連続写像 (岩井雅崇, 2020/10/06)}
%\author{岩井雅崇}
%\date{2020/10/06}
%ここまで今回の記事関係ない
\usepackage{tcolorbox}
\tcbuselibrary{breakable, skins, theorems}

\theoremstyle{definition}
\newtheorem{thm}{定理}
\newtheorem{lem}[thm]{補題}
\newtheorem{prop}[thm]{命題}
\newtheorem{cor}[thm]{系}
\newtheorem{claim}[thm]{主張}
\newtheorem{dfn}[thm]{定義}
\newtheorem{rem}[thm]{注意}
\newtheorem{exa}[thm]{例}
\newtheorem{conj}[thm]{予想}
\newtheorem{prob}[thm]{問題}
\newtheorem{rema}[thm]{補足}

\DeclareMathOperator{\Ric}{Ric}
\DeclareMathOperator{\Vol}{Vol}
 \newcommand{\pdrv}[2]{\frac{\partial #1}{\partial #2}}
 \newcommand{\drv}[2]{\frac{d #1}{d#2}}
  \newcommand{\ppdrv}[3]{\frac{\partial #1}{\partial #2 \partial #3}}


%ここから本文.
\begin{document}

\begin{center}
{\Large 第8回. リーマン積分の定義と微分積分学の基本定理 \\ (三宅先生の本, 3.1と3.4の内容)}
\end{center}

\begin{flushright}
 岩井雅崇 2021/06/08
\end{flushright}


\section{はじめに}
 この回の内容はかなり難しいので, 積分の理論を気にせず計算だけしたい人はこの回の内容を読み飛ばして, 次の回の内容に移って良い.
 %(最後の誤差評価は使えるかもしれませんが...)
 また証明等を少々省略するので, 詳しくリーマン積分を理解したい人は「杉浦光夫 解析入門 1 (東京大学出版会)」を見てほしい.
% \begin{itemize}\item 杉浦光夫 解析入門 1 (東京大学出版会)\end{itemize}
 
%動画冒頭に述べたルベーグ積分を理解したい人は次の文献を見てほしい.(めちゃくちゃ難しいですが...) \begin{itemize} \item 伊藤清三 ルベーグ積分入門 (裳華房) \item Terence Tao \textit{An introduction to measure theory} available at \url{https://terrytao.files.wordpress.com/2011/01/measure-book1.pdf} \end{itemize}リーマン積分もルベーグ積分もどちらも計算上は違いはないので, 積分の理論を気にせず, 計算だけしたい場合は気にしなくて良いです.


\section{リーマン積分の定義}
この節では$I = [a,b]$とし, 関数$f(x)$は$I$上で有界であるとする.
 \begin{itemize}
 \item \underline{$\Delta$が$I$の分割}とは, 正の自然数$m$と
 $
 a = x_{0}<x_1< \dots , x_{m-1}<x_{m}=b %\text{\,\,となる}
 $
となる数の組$( a, x_1, \dots , x_{m-1} , b) $のこと.
 以下$\Delta = ( a, x_1, \dots , x_{m-1} , b ) $とかく. (この授業だけの記号である.)
 \item $\Delta$を$I$の分割として, \underline{$\Delta$の長さ}を
 $
| \Delta| = \max_{1 \leqq i \leqq m, } \{ |x_i - x_{i-1}| \} 
 \text{\,\,とする.}
 $
 
 \item $\Delta$を$I$の分割とする.
 $1 \leqq i \leqq m$となる自然数$i$について
 $$
 M_{i} = \sup \{ f(x) \,\,| \,\,x_{i-1} \leqq x \leqq x_i \}, \text{\,\,\,\,}
 m_{i} = \inf \{ f(x) \,\,| \,\,x_{i-1} \leqq x \leqq x_i  \} \text{\,\,とし, }
$$
 $$
 S_{\Delta} = \sum_{i=1}^{m} M_{i}(x_i - x_{i-1}), \text{\,\,\,\,}
  T_{\Delta} =\sum_{i=1}^{m} m_{i}(x_i - x_{i-1})\text{\,\,とおく. }
 $$
定義から$T_{\Delta} \leqq S_{\Delta}$となる.

 \end{itemize}
 
  \begin{tcolorbox}[
    colback = white,
    colframe = green!35!black,
    fonttitle = \bfseries,
    breakable = true]
    \begin{thm}[ダルブーの定理]
    ある実数$S,T$があって, 
    $$
    \lim_{|\Delta| \rightarrow 0}S_{\Delta} = S, \text{\,\,} \lim_{|\Delta| \rightarrow 0}T_{\Delta} = T.
    $$
    \end{thm}
    \end{tcolorbox}
    \footnote{$\lim_{|\Delta| \rightarrow 0}S_{\Delta} = S$の意味は, $\Delta$の長さが0になるように分割をとっていくと, $S_{\Delta}$は$S$に限りなく近くという意味である.}
    
      \begin{tcolorbox}[
    colback = white,
    colframe = green!35!black,
    fonttitle = \bfseries,
    breakable = true]
    \begin{dfn}[リーマン積分の定義]
    $I = [a,b]$かつ$f(x)$を$I$上の有界関数とする. \\
    \underline{$f$が$I$上でリーマン積分可能(リーマン可積分)}とは$S=T$となること.
    このとき, 
    $$
    S = \int_{a}^{b} f(x)dx \text{\,\,と表す.}
    $$
    \underline{$\int_{a}^{b} f(x)dx $を$f(x)$の$[a,b]$における定積分}という.
    
    また
    $$
    \int_{a}^{a} f(x)dx  =0, \text{\,\,\,\,} \int_{b}^{a} f(x)dx = -\int_{a}^{b} f(x)dx 
    \text{\,\,とする.}$$
    \end{dfn}
    \end{tcolorbox}
    以下, リーマン積分可能を単に積分可能ということにする.

\begin{exa}
\label{riem_not}
\begin{itemize}
\item $I = [a,b]$とし, $f$を$I$上での連続関数とする.
このとき$f$は$I$上で積分可能.(みんながよく知っている関数は積分可能.)
\item $I = [0,1]$とし, $I$上の有界関数$f(x)$を
$$
  f(x)= \begin{cases}
     1& \text{$x$は有理数}\\
    0& \text{$x$は無理数}
  \end{cases}
$$
とおくとき, 任意の$I$の分割$\Delta$について, $S_{\Delta}=1$であり, $T_{\Delta}=0$である.
よって$S =1$かつ$T=0$より, $f$は$I$上で積分可能ではない.
%\footnote{ルベーグ積分は可能になる. 積分値は0となる. ルベーグ積分はいい感じにリーマン積分を包含する概念である.}
 \end{itemize}
\end{exa}

      \begin{tcolorbox}[
    colback = white,
    colframe = green!35!black,
    fonttitle = \bfseries,
    breakable = true]
    \begin{thm}[区分求積法]
$I=[a,b]$とし, $f(x)$を$I$上の積分可能な関数とする.
任意の$n\in N$について, 
$x_i = a + \frac{(b-a)i}{n}$
($i$は$1 \leqq i \leqq n$なる自然数)とおき, 
$
D_n = \sum_{i=1}^{n} \frac{f(x_i)}{n} %\text{\,\,\ とすると}
$
とすると, 
$$
 \lim_{n \rightarrow \infty }D_n = \int_{a}^{b} f(x) dx\text{\,\,\ となる.}
$$
        \end{thm}
    \end{tcolorbox}
 \begin{exa}
 $I =[0,1]$とし, $f(x) =x^2$を$I$上の関数とする.
 任意の$n\in N$について, 
$x_i = 0+ \frac{(1-0)i}{n} = \frac{i}{n}$($i$は$1 \leqq i \leqq n$なる自然数)であるので, 
$$
D_n =  \sum_{i=1}^{n} \frac{f(x_i)}{n} 
= \sum_{i=1}^{n} \frac{i^2}{n^3} = \frac{n(n+1)(2n+1)}{6n^3}
$$
以上より区分求積法から
$$
\int_{0}^{1} x^2 dx =
 \lim_{n \rightarrow \infty }D_n = \frac{1}{3}\text{\,\,\ である.}
$$
 \end{exa}

    
\section{微分積分学の基本定理}

      \begin{tcolorbox}[
    colback = white,
    colframe = green!35!black,
    fonttitle = \bfseries,
    breakable = true]
    \begin{dfn}
    $f(x)$を区間$I$上の連続関数とする.
    $a \in I$を一つ固定する.
     $$
\begin{array}{cccc}
F: &I& \rightarrow & \R  \\
&x& \longmapsto & \int_{a}^{x} f(x) dx
\end{array}
$$
を\underline{$f(x)$の不定積分}と呼ぶ. 
\underline{$F(x)$を$\int f(x) dx$とも表す.}
不定積分は定数を除いてただ一つに定まる.
        \end{dfn}
    \end{tcolorbox}
      \begin{tcolorbox}[
    colback = white,
    colframe = green!35!black,
    fonttitle = \bfseries,
    breakable = true]
    \begin{thm}[微分積分学の基本定理]
    $f(x)$を区間$I$上の連続関数とする.
不定積分$F(x) = \int_{a}^{x} f(x) dx$は$I$上で微分可能で
$F'(x)=f(x)$である.特に
$$
\drv{}{x} \int_{a}^{x} f(t)dt = f(x) \text{\,\,\,である.}
$$
        \end{thm}
    \end{tcolorbox}
    
\begin{tcolorbox}[
    colback = white,
    colframe = green!35!black,
    fonttitle = \bfseries,
    breakable = true]
    \begin{prop}
    $f(x)$を区間$I$上の連続関数とし, $G(x)$を$I$上の関数とする.
    $G'(x) = f(x)$ならばある定数$c$があって, 
    $$
    G(x) = \int f(x) dx + c \text{\,\,\,となる.}
    $$
        \end{prop}
    \end{tcolorbox}
\begin{exa}
$f(x) = x^2, G(x) = \frac{x^3}{3}$とすると
$G'(x) = \left( \frac{x^3}{3} \right)' = x^2=f(x)$よりある定数$c$があって
$ \int x^2 dx   = \frac{x^3}{3} + c$となる.

\end{exa}
    
 
\begin{tcolorbox}[
    colback = white,
    colframe = green!35!black,
    fonttitle = \bfseries,
    breakable = true]
    \begin{thm}
    $f(x)$を$[a,b]$上の連続関数とし, $G(x)$を$G'(x) = f(x)$となる$[a,b]$上の関数とする.
このとき
$$
\int_{a}^{b} f(x) dx = \Bigl[ G(x) \Bigr]_{a}^{b} = G(b) - G(a) \text{\,\,\,となる.}
$$
        \end{thm}
    \end{tcolorbox}

\begin{exa}
$$\int^{1}_{0} x^2 dx = \left[\frac{x^3}{3}\right]^{1}_{0} =\frac{1}{3} $$である.
(区分求積法を用いるよりもずっとずっと簡単である.)
\end{exa}
    
 
\section{演習問題}
演習問題の解答は授業の黒板にあります.
\begin{enumerate}
\item[] $n,x \in \N$について
$$S_{n}(x) = \sum_{k=1}^{n}\frac{1}{k^x}\text{\,\,\,とする.}$$
次の問いに答えよ.
\item $\lim_{n \rightarrow \infty} S_{n}(1)$は発散することを示せ.
\item $\lim_{n \rightarrow \infty} S_{n}(2)$は収束することを示せ.
\end{enumerate}
ちなみに
\begin{align*}
\begin{split}
\frac{\pi^2}{6} &= \lim_{n \rightarrow \infty} S_{n}(2) = 1 + \frac{1}{4}+ \frac{1}{9}+ \frac{1}{16}+ \frac{1}{25}+
\cdots  \\
\frac{\pi^4}{90} &= \lim_{n \rightarrow \infty} S_{n}(4) = 1 + \frac{1}{2^4}+ \frac{1}{3^4}+ \frac{1}{4^4}+ \frac{1}{5^4}+
\cdots  \\
\frac{\pi^6}{945} &= \lim_{n \rightarrow \infty} S_{n}(6) = 1 + \frac{1}{2^6}+ \frac{1}{3^6}+ \frac{1}{4^6}+ \frac{1}{5^6}+
\cdots  \\
\end{split}
\end{align*}
であることが知られている.
 
\newpage

\begin{center}
{\Large 第9回. 積分の性質 (三宅先生の本, 3.1と3.2の内容)}
\end{center}

\begin{flushright}
 岩井雅崇 2021/06/15
\end{flushright}


\section{積分の性質}
 
\begin{tcolorbox}[
    colback = white,
    colframe = green!35!black,
    fonttitle = \bfseries,
    breakable = true]
    \begin{thm}
    $f(x)$を$[a,b]$上の連続関数とし, $F(x)$を$F'(x) = f(x)$となる$[a,b]$上の関数とする.
このとき
$$
\int_{a}^{b} f(x) dx = \Bigl[ F(x) \Bigr]_{a}^{b} = F(b) - F(a) \text{\,\,\,となる.}
$$
        \end{thm}
    \end{tcolorbox}
    \begin{tcolorbox}[
    colback = white,
    colframe = green!35!black,
    fonttitle = \bfseries,
    breakable = true]
    \begin{prop}[積分の性質]
$f(x), g(x)$共に$[a,b]$上の連続関数とし, $G(x) = \int g(x) dx$とする.
\begin{enumerate}
\item $\int_{a}^{b} (f(x) \pm g(x)) dx = \int_{a}^{b} f(x) dx \pm \int_{a}^{b} g(x) dx$
\item $k$を定数とするとき, $\int_{a}^{b} kf(x) dx  = k \int_{a}^{b} f(x) dx $
\item (置換積分法)    $$
\begin{array}{cccc}
x(t): &[\alpha, \beta]& \rightarrow & [a,b]\\
&t& \longmapsto & x(t)
\end{array}
$$
を$C^1$級関数とし, $a = x(\alpha), b=x(\beta)$とするとき
$$
\int_{a}^{b} f(x) dx = \int_{\alpha}^{\beta} f(x(t)) \drv{x(t)}{t} dt \text{\,\,\,となる.}
$$
\item (部分積分法)  $f(x)$が$C^1$級であるとき,
$$
\int_{a}^{b} f(x) g(x)dx = \Bigl[ f(x) G(x)\Bigr]^{b}_{a} - \int_{a}^{b}f'(x) G(x)dx
\text{\,\,\,となる.}$$
\end{enumerate}

        \end{prop}
    \end{tcolorbox}

\section{不定積分の例}
簡単な積分に関してまとめておく. %(以下は覚えなくても良い.) 
積分定数に関しては省略する. また$a$を実数とする.
  \begin{align*}
\begin{split}
\int x^a \,dx &= \frac{x^{a+1}}{a+1} \text{\,\,\, ($a \neq -1$のとき)}\\
\int \frac{1}{x} \,dx &= \log |x|\\
\int \frac{1}{\sqrt{a^2 - x^2}} \,dx &= \Sin \frac{x}{|a|} \text{\,\,\, ($a \neq 0$のとき)}\\
\int \frac{1}{a^2 + x^2} \,dx &= \frac{1}{a}\Tan \frac{x}{a} \text{\,\,\, ($a \neq 0$のとき)}\\
\int e^x \,dx &=e^x\\
\int a^x \,dx &=\frac{1}{\log a}a^x \text{\,\,\, ($a>0$かつ$a \neq 1$のとき)} \\
\int \log x  \,dx &=x \log x -x\\
\int \sin x  \,dx &=-\cos x \\
\int \cos x  \,dx &=\sin x \\
\int \frac{1}{(\cos x)^2}  \,dx &=\tan x \\
\end{split}
\end{align*}
 
\section{ウォリスの公式}

\begin{tcolorbox}[
    colback = white,
    colframe = green!35!black,
    fonttitle = \bfseries,
    breakable = true]
    \begin{thm}
 $n$を自然数として, 
 $$I_n = \int_{0}^{\frac{\pi}{2}} (\cos t)^n dt =\int_{0}^{\frac{\pi}{2}} (\sin t)^n dt \text{\,\,とする.}$$
$n$が偶数のとき, 
$$
 I_n =  
 \frac{(n-1)!!}{n!!}\frac{\pi}{2} 
 =
 \frac{n-1}{n}\cdot\frac{n-3}{n-2} \cdots \frac{3}{4}\cdot\frac{1}{2}\cdot \frac{\pi}{2}\text{\,\,であり.}
$$
$n$が奇数のとき, 
$$
 I_n =  \frac{(n-1)!!}{n!!} 
 =
 \frac{n-1}{n}\cdot\frac{n-3}{n-2} \cdots \frac{4}{5}\cdot\frac{2}{3} \cdot 1 \text{\,\,である.}
 $$
        \end{thm}
    \end{tcolorbox}
    \footnote{$n!!$は二重階乗と呼ばれる. $n$を正の自然数として, $(2n-1)!!=(2n-1)(2n-3) \cdots 3\cdot1 $, $(2n)!!=(2n)(2n-2) \cdots 4\cdot2 $である. 便宜上$0!!=1$とする.($0!=1$であるので.)}
    
 \begin{tcolorbox}[
    colback = white,
    colframe = green!35!black,
    fonttitle = \bfseries,
    breakable = true]
    \begin{thm}[ウォリスの公式]
  \begin{align*}
  \begin{split}
\frac{\pi}{2} 
&=
 \lim_{m\rightarrow \infty} \frac{(2m)^2}{ (2m+1)(2m-1)} \cdot 
 \frac{(2(m-1))^2}{ (2m-1)(2m-3)} \cdots \frac{2^2}{3\cdot1 }  \\
& =
  \lim_{m\rightarrow \infty}\frac{1}{(1- \frac{1}{4m^2})} \cdot \frac{1}{(1- \frac{1}{4(m-1)^2})}
  \cdots \frac{1}{(1- \frac{1}{4})} \text{\,\,となる.}
  \end{split}
   \end{align*}
つまり
$$
\frac{\pi}{2} 
=
\frac{2 \cdot 2}{1 \cdot 3} \cdot \frac{4 \cdot 4}{3 \cdot 5} 
\cdot \frac{6 \cdot 6}{5 \cdot 7}\cdot \frac{8 \cdot 8}{7 \cdot 9}
 \cdots  \text{\,\, である.}
$$
   
        \end{thm}
    \end{tcolorbox}
         \footnote{ 積の記号を使って書けば, $$\frac{\pi}{2} =\lim_{m\rightarrow \infty}\frac{1}{(1- \frac{1}{4m^2})} \cdot \frac{1}{(1- \frac{1}{4(m-1)^2})}\cdots \frac{1}{(1- \frac{1}{4})} = \Pi_{i=0}^{\infty} \frac{1}{(1- \frac{1}{4i^2})}\text{\,\, である.}$$}
 

\section{演習問題}
演習問題の解答は授業の黒板にあります.
\begin{enumerate}
\item 不定積分$\int x\log x \,dx$を求めよ.
%\item 不定積分$\int \log x \,dx$を求めよ
%\item 定積分$\int_{0}^{1} x  e^{2x} \,dx$を求めよ.
\end{enumerate}



\newpage

\begin{center}
{\Large 第10回. 不定積分の計算方法 (三宅先生の本, 3.2の内容)}
\end{center}

\begin{flushright}
 岩井雅崇 2021/06/22
\end{flushright}


\section{不定積分の計算方法・テクニック}
\subsection{有理式の場合}
 
\begin{tcolorbox}[
    colback = white,
    colframe = green!35!black,
    fonttitle = \bfseries,
    breakable = true]
    \begin{dfn}[有理式]
  $f(x)$と$g(x)$を実数係数多項式とするとき, \underline{$\frac{f(x)}{g(x)}$を有理式}という.
        \end{dfn}
    \end{tcolorbox}

以下$f(x)$と$g(x)$を同時に割り切る多項式はないものと仮定する.(つまり$f(x)$と$g(x)$は互いに素とする.)

\begin{tcolorbox}[
    colback = white,
    colframe = green!35!black,
    fonttitle = \bfseries,
    breakable = true]
    \begin{thm}[有理式]
有理式$\frac{f(x)}{g(x)}$は次の3つの式の和に分解できる.
\begin{enumerate}
\item 多項式
\item $\frac{a}{(x+b)^m}$ ($a,b \in \R, m\in \N$)
\item $\frac{ax + b}{(x^2 + cx +d)^m}$ ($a,b,c,d \in \R, m\in \N$)
\end{enumerate}
特に$\alpha_1, \ldots, \alpha_l \in \R$と$m_1, \ldots, m_l \in \N$を用いて
$g(x) = (x- \alpha_1)^{m_1} \cdots (x- \alpha_l)^{m_l} $と書けるとき, 
有理式$\frac{f(x)}{g(x)}$は多項式と$\frac{\beta_i}{(x- \alpha_i)^m}$ ($\beta_i \in \R, m\in \N$, $1 \leqq m\leqq m_i$)の和で表せられる.
        \end{thm}
    \end{tcolorbox}
\begin{exa}
$\frac{5x -4 }{2x^2 + x -6}$に関して上の定理より, 
$$
\frac{5x -4 }{2x^2 + x -6} = \frac{a}{2x-3} + \frac{b}{x+2}
$$
となる実数$a,b \in \R$が存在する.
通分して計算すると$a=1, b=2$をえる.

\end{exa}
\begin{exa}
$$
\frac{2}{(x-1)(x^2 + 1)} = \frac{1}{x-1} - \frac{x+1}{x^2 + 1}
$$
\end{exa}

\begin{tcolorbox}[
    colback = white,
    colframe = green!35!black,
    fonttitle = \bfseries,
    breakable = true]
    \begin{thm}
有理式の不定積分は計算できる.
        \end{thm}
    \end{tcolorbox}
\begin{exa}
$$
\int \frac{5x -4 }{2x^2 + x -6}dx = \int \frac{1}{2x-3} dx+ \int \frac{2}{x+2}dx
= \frac{1}{2}\log|2x-3| + 2 \log |x+2|
$$
$$
\int \frac{2}{(x-1)(x^2 + 1)} dx = \int \frac{1}{x-1}dx  - \int \frac{x+1}{x^2 + 1}dx
= \log|x-1| - \frac{1}{2} \log |x^2+1|- \Tan x
$$
\end{exa}

\subsection{無理関数がある場合}
テクニックだけまとめておく.
\begin{itemize}
\item $\sqrt[n]{ax + b}$に関して, $t= \sqrt[n]{ax + b}$とおくと
$$
x = \frac{t^n - b}{a}, dx = \frac{n t^{n-1} dt}{a}
$$
より有理式に帰着できる.
\item $\sqrt{ax^2 + b x + c}$に関して, $a>0$ならば$\sqrt{ax^2 + b x + c} = t - \sqrt{a} x$とおく.
\item $\sqrt{ax^2 + b x + c}$に関して, $ax^2 + b x + c = (x - \alpha)(x- \beta)$となる実数$\alpha, \beta \in \R$があるとき, $t = \sqrt{\frac{a(x - \beta)}{(x-\alpha)}}$とおく.
\end{itemize}

\begin{exa}
不定積分$\int \frac{dx}{x + 2 \sqrt{x-1}}$を求めよ.

\hspace{-18pt}(答.) $t = \sqrt{x-1}$とおくと, $x = t^2 +1, dx = 2tdt$より
\begin{align*}
\begin{split}
\int \frac{dx}{x + 2 \sqrt{x-1}}
&= \int \frac{2t}{t^2 + 1 + 2t} dt= \int \frac{2t + 2 -2}{(t + 1)^2} dt \\
&=  \int \frac{2}{t + 1} dt -  \int \frac{2}{(t + 1)^2} dt \\
&= 2 \log |t+1| + \frac{2}{t+1} \\
&= 2 \log (1 + \sqrt{x-1}) + \frac{2}{1 + \sqrt{x-1}}
\end{split}
\end{align*}

\end{exa}

\subsection{三角関数の有理式の積分}

\begin{tcolorbox}[
    colback = white,
    colframe = green!35!black,
    fonttitle = \bfseries,
    breakable = true]
    \begin{thm}
三角関数に関する有理式の不定積分は計算できる.
具体的には$t = \tan \frac{x}{2}$とおけば, $\sin x$などは次のように表される.
\begin{itemize}
\item $dx = \frac{2 dt}{1+ t^2}$
\item $\sin x = \frac{2t}{1+ t^2}$
\item $\cos x = \frac{1 - t^2}{1+ t^2}$
\item $\tan x = \frac{2t}{1- t^2}$
\end{itemize}


        \end{thm}
    \end{tcolorbox}
    
\begin{exa}
不定積分$\int \frac{1 + \sin x}{1 + \cos x} dx$を求めよ.

\hspace{-18pt}(答.) $t = \tan \frac{x}{2}$とおくと, 
\begin{align*}
\begin{split}
\int \frac{1 + \sin x}{1 + \cos x} dx
&= \int \frac{1 + \frac{2t}{1+t^2} }{ 1+ \frac{1-t^2}{1+ t^2}} 
\frac{2dt}{1+ t^2} = \int \frac{t^2 + 2t + 1}{1+ t^2} dt = \int 1 +\frac{2t}{1+ t^2} dt \\
&= t + \log (1 + t^2) = \tan \frac{x}{2} + \log \left|1 + \left(\tan \frac{x}{2} \right)^2 \right|\\
\end{split}
\end{align*}

\end{exa}


\section{演習問題}
演習問題の解答は授業の黒板にあります.
\begin{enumerate}
\item 不定積分$\int \frac{x^2}{x^2 - x - 6}dx$を求めよ.
\end{enumerate}


\newpage

\begin{center}
{\Large 第11回. 広義積分 (三宅先生の本, 3.3の内容)}
\end{center}

\begin{flushright}
 岩井雅崇 2021/06/29
\end{flushright}



\section{広義積分}

 \begin{tcolorbox}[
    colback = white,
    colframe = green!35!black,
    fonttitle = \bfseries,
    breakable = true]
    \begin{dfn}[広義積分]
 $a$を実数とし, $b$は実数または$b=+\infty$とする. $f(x)$を$[a,b)$上の連続関数とする.
左極限$\lim_{\beta \rightarrow b-0} \int_{a}^{\beta} f(x)dx$が存在するとき, 
\underline{広義積分 $\int_{a}^{b} f(x)dx$は収束する}といい
$$
\int_{a}^{b} f(x)dx = \lim_{\beta \rightarrow b-0} \int_{a}^{\beta} f(x)dx \text{\,\,とする.}
$$
この積分を\underline{広義積分}という.
極限が存在しないときは, \underline{広義積分$\int_{a}^{b} f(x)dx$は発散する}という.
 \end{dfn}
 \end{tcolorbox}
 

 \begin{exa}
  \begin{itemize}
 \item $\int_{1}^{\infty} x^p dx$は$p<-1$のとき収束し, $p \geqq -1$のとき発散する.
 \item $\int_{0}^{1} x^p dx$は$p>-1$のとき収束し, $p \leqq -1$のとき発散する.
\end{itemize}
 \end{exa}


  \begin{tcolorbox}[
    colback = white,
    colframe = green!35!black,
    fonttitle = \bfseries,
    breakable = true]
    \begin{thm}
$f(x)$を$[a,b)$上の連続関数とする.
$[a,b)$上の連続関数$g(x)$があって, $[a , b)$上で$|f(x)| \leqq g(x)$かつ広義積分$\int_{a}^{b} g(x) dx$が収束すると仮定する.
このとき広義積分$\int_{a}^{b} f(x) dx$もまた収束する.
 \end{thm}
 \end{tcolorbox}
 
   \begin{tcolorbox}[
    colback = white,
    colframe = green!35!black,
    fonttitle = \bfseries,
    breakable = true]
    \begin{thm}
$f(x)$を$[a,b)$上の連続関数とする.
$[a,b)$上の連続関数$g(x)$があって, $[a , b)$上で$0 \leqq g(x) \leqq f(x)$かつ広義積分$\int_{a}^{b} g(x) dx$が発散すると仮定する.
このとき広義積分$\int_{a}^{b} f(x) dx$もまた発散する.
 \end{thm}
 \end{tcolorbox}
 
 \begin{exa}
広義積分$\int_{0}^{1} \frac{\sin x}{\sqrt{1-x}}dx $は収束する.
これは$[0, 1)$上で$|\frac{\sin x}{\sqrt{1-x}}| \leqq \frac{1}{\sqrt{1-x}}$かつ
広義積分$\int_{0}^1 \frac{1}{\sqrt{1-x}} dx$が収束するからである.
 \end{exa}
 
  \begin{exa}
広義積分$\int_{2}^{\infty} \frac{1}{\sqrt[3]{x(x-1)}}dx $は発散する.
これは$[2, \infty)$上で
$0 \leqq x^{-\frac{2}{3}}\leqq \frac{1}{\sqrt[3]{x(x-1)}}$
かつ
広義積分$\int_{2}^{\infty} x^{- \frac{2}{3}} dx$が発散するからである.
 \end{exa}

  \begin{exa}
実数$s>0$について, 広義積分$\int_{0}^{\infty} e^{-x}x^{s-1}dx $は収束する.

\hspace{-18pt}(証.)
$\lim_{x \rightarrow \infty} (e^{-x}x^{s-1}) x^{2} = \lim_{x \rightarrow \infty} e^{-x}x^{s+1} =0$
より, ある$c>0$があって, $[c +\infty)$上で$e^{-x} x^{s-1} \leqq x^{-2}$である.
広義積分$\int_{c}^{\infty} x^{-2} dx$は収束するため, 広義積分$\int_{c}^{\infty}e^{-x} x^{s-1}dx$も収束する.

一方$(0 , c]$上で$e^{-x}x^{s-1} \leqq x^{s-1}$であり, $s-1 > -1$から広義積分$\int_{0}^{c} x^{s-1} dx$は収束するため
広義積分$\int_{0}^{c} e^{-x}x^{s-1} dx$も収束する.

以上より広義積分$\int_{0}^{\infty} e^{-x}x^{s-1}dx  =\int_{0}^{c} e^{-x}x^{s-1}dx +\int_{c}^{\infty} e^{-x}x^{s-1}dx  $は収束する.
 \end{exa}



\section{演習問題}
演習問題の解答は授業の黒板にあります.
\begin{enumerate}
\item[] $p$を実数とし$f(x) = x^p \log x$とする.
\item $p< -1$ならば広義積分$\int_{1}^{\infty} f(x) dx$は収束することを示せ.
\item $p\geqq -1$ならば広義積分$\int_{1}^{\infty} f(x) dx$は発散することを示せ.
\end{enumerate}



\newpage

\begin{center}
{\Large 第12回. 曲線の長さ (三宅先生の本, 3.4の内容)}
\end{center}

\begin{flushright}
 岩井雅崇 2021/07/06
\end{flushright}



\section{曲線の定義と曲線の長さ}
 \begin{tcolorbox}[
    colback = white,
    colframe = green!35!black,
    fonttitle = \bfseries,
    breakable = true]
    \begin{dfn}
 %   \text{}\begin{itemize}\item 
%関数 $\vec{p}(t)$ を次で定める.
$$
\begin{array}{ccccc}
C: &[a,b] & \rightarrow & \R^2 & \\
&t & \longmapsto &(x(t), y(t))&
\end{array}
$$
が \underline{滑らかな曲線}とは次の2条件を満たすこと.
\begin{itemize}
\item $x(t),y(t)$共に$[a,b]$上の$C^1$級関数.
\item 任意の$t \in (a,b)$について, 速度ベクトル$ (x'(t), y'(t))\neq  (0,0)$である.
\end{itemize}
%\item \underline{曲線$C: \vec{p}(t) (a \leqq t \leqq b)$が区分的に滑らかな曲線}とは滑らかな曲線を端点でつないだもの.\end{itemize}

滑らかな曲線$C$に関してその長さを
$$
l(C) = \int_{a}^{b} \sqrt{(x'(t))^2 + (y'(t))^2} dt \text{\,\,\, とする.}
$$
     \end{dfn}
 \end{tcolorbox}
 
  \begin{tcolorbox}[
    colback = white,
    colframe = green!35!black,
    fonttitle = \bfseries,
    breakable = true]
    \begin{thm}
$f(x)$を$[a,b]$上の$C^1$級関数とする. このとき$y=f(x)$のグラフ$C=\{ (x, f(x))\,\,| \,\, a \leqq x \leqq b\}$の長さは
$$
l(C) = \int_{a}^{b} \sqrt{1 + (f'(x))^2} dx \text{\,\,\, である.}
$$
     \end{thm}
 \end{tcolorbox}
 
 \begin{exa}
 放物線$y=x^2 (0 \leqq x \leqq 1)$のグラフの長さを求めよ.
 
\hspace{-18pt} (答.) $f(x) = x^2$とすると$f'(x) = 2x$のため, 曲線の長さは\begin{align*}
\begin{split}
\int_{0}^{1} \sqrt{1 + (2x)^2} dx &= 
\int_{0}^{1} \sqrt{1 +4x^2} dx = 
\frac{1}{2}\int_{0}^{2} \sqrt{1 +t^2} dt \\
&= \frac{1}{4}\left[ t\sqrt{t^2 + 1} + \log \Bigl| t + \sqrt{t^2 + 1 }\Bigr| \right]^{2}_{0} \\
&= \frac{1}{4} \left(2 \sqrt{5} + \log (2 + \sqrt{5})\right)
\end{split}
\end{align*}
 \end{exa}

  \begin{tcolorbox}[
    colback = white,
    colframe = green!35!black,
    fonttitle = \bfseries,
    breakable = true]
    \begin{thm}
$[\alpha,\beta]$上の$C^1$級関数$f( \theta )$を用いて, 曲線$C$が
$$
\begin{array}{ccccc}
C: &[\alpha, \beta] & \rightarrow & \R^2 & \\
&\theta & \longmapsto &(f(\theta) \cos \theta, f(\theta) \sin \theta)&
\end{array}
$$
%$$C: (x(\theta), y(\theta)) = (f(\theta) \cos \theta, f(\theta) \sin \theta)$$
と表されているとき, $C$の長さは
$$
 \int_{\alpha}^{\beta} \sqrt{(f(\theta))^2 + (f'(\theta))^2} d\theta \text{\,\,\, である.}
$$
     \end{thm}
 \end{tcolorbox}
 
 \begin{exa}
(アルキメデスの螺旋)
正の実数$a , \alpha $について
$$
\begin{array}{ccccc}
C: &[0,\alpha] & \rightarrow & \R^2 & \\
&\theta & \longmapsto &(a \theta \cos \theta ,  a \theta \sin \theta)&
\end{array}
$$
 とする. 曲線$C$の長さを求めよ.
 
\hspace{-18pt} (答.) $f(\theta) = a \theta$とすると$f'(\theta) = a$のため, 曲線の長さは\begin{align*}
\begin{split}
\int_{0}^{\alpha} \sqrt{a^2 + (a \theta)^2} d\theta = 
a \int_{0}^{\alpha} \sqrt{1+ (\theta)^2} d\theta = 
\frac{a}{2}\left(\alpha\sqrt{\alpha^2 + 1} + \log (\alpha + \sqrt{\alpha^2 + 1})\right)
\end{split}
\end{align*}
 \end{exa}
 
\section{演習問題}
演習問題の解答は授業の黒板にあります.
\begin{enumerate}
\item 正の実数$a,b$について$f(x) = a \cosh \frac{x}{a} -a$とする.
グラフ$C=\{ (x, f(x))\,\,| \,\, 0 \leqq x \leqq b\}$の長さを$a,b$を用いて表せ.
\end{enumerate}





\end{document}
