\documentclass[dvipdfmx,a4paper,11pt]{article}
\usepackage[utf8]{inputenc}
%\usepackage[dvipdfmx]{hyperref} %リンクを有効にする
\usepackage{url} %同上
\usepackage{amsmath,amssymb} %もちろん
\usepackage{amsfonts,amsthm,mathtools} %もちろん
\usepackage{braket,physics} %あると便利なやつ
\usepackage{bm} %ラプラシアンで使った
\usepackage[top=30truemm,bottom=30truemm,left=25truemm,right=25truemm]{geometry} %余白設定
\usepackage{latexsym} %ごくたまに必要になる
\renewcommand{\kanjifamilydefault}{\gtdefault}
\usepackage{otf} %宗教上の理由でmin10が嫌いなので


\usepackage[all]{xy}
\usepackage{amsthm,amsmath,amssymb,comment}
\usepackage{amsmath}    % \UTF{00E6}\UTF{0095}°\UTF{00E5}\UTF{00AD}\UTF{00A6}\UTF{00E7}\UTF{0094}¨
\usepackage{amssymb}  
\usepackage{color}
\usepackage{amscd}
\usepackage{amsthm}  
\usepackage{wrapfig}
\usepackage{comment}	
\usepackage{graphicx}
\usepackage{setspace}
\setstretch{1.2}


\newcommand{\R}{\mathbb{R}}
\newcommand{\Z}{\mathbb{Z}}
\newcommand{\Q}{\mathbb{Q}} 
\newcommand{\N}{\mathbb{N}}
\newcommand{\C}{\mathbb{C}} 
\newcommand{\Sin}{\text{Sin}^{-1}} 
\newcommand{\Cos}{\text{Cos}^{-1}} 
\newcommand{\Tan}{\text{Tan}^{-1}} 
\newcommand{\invsin}{\text{Sin}^{-1}} 
\newcommand{\invcos}{\text{Cos}^{-1}} 
\newcommand{\invtan}{\text{Tan}^{-1}} 
\newcommand{\Area}{\text{Area}}
\newcommand{\vol}{\text{Vol}}




   %当然のようにやる.
\allowdisplaybreaks[4]
   %もちろん.
%\title{第1回. 多変数の連続写像 (岩井雅崇, 2020/10/06)}
%\author{岩井雅崇}
%\date{2020/10/06}
%ここまで今回の記事関係ない
\usepackage{tcolorbox}
\tcbuselibrary{breakable, skins, theorems}

\theoremstyle{definition}
\newtheorem{thm}{定理}
\newtheorem{lem}[thm]{補題}
\newtheorem{prop}[thm]{命題}
\newtheorem{cor}[thm]{系}
\newtheorem{claim}[thm]{主張}
\newtheorem{dfn}[thm]{定義}
\newtheorem{rem}[thm]{注意}
\newtheorem{exa}[thm]{例}
\newtheorem{conj}[thm]{予想}
\newtheorem{prob}[thm]{問題}
\newtheorem{rema}[thm]{補足}

\DeclareMathOperator{\Ric}{Ric}
\DeclareMathOperator{\Vol}{Vol}
 \newcommand{\pdrv}[2]{\frac{\partial #1}{\partial #2}}
 \newcommand{\drv}[2]{\frac{d #1}{d#2}}
  \newcommand{\ppdrv}[3]{\frac{\partial #1}{\partial #2 \partial #3}}


%ここから本文.
\begin{document}
%\maketitle
\begin{center}
{\Large 第12回. 曲線の長さ (三宅先生の本, 3.4の内容)}
\end{center}

\begin{flushright}
 岩井雅崇 2021/07/06
\end{flushright}



\section{曲線の定義と曲線の長さ}
 \begin{tcolorbox}[
    colback = white,
    colframe = green!35!black,
    fonttitle = \bfseries,
    breakable = true]
    \begin{dfn}
 %   \text{}\begin{itemize}\item 
%関数 $\vec{p}(t)$ を次で定める.
$$
\begin{array}{ccccc}
C: &[a,b] & \rightarrow & \R^2 & \\
&t & \longmapsto &(x(t), y(t))&
\end{array}
$$
が \underline{滑らかな曲線}とは次の2条件を満たすこと.
\begin{itemize}
\item $x(t),y(t)$共に$[a,b]$上の$C^1$級関数.
\item 任意の$t \in (a,b)$について, 速度ベクトル$ (x'(t), y'(t))\neq  (0,0)$である.
\end{itemize}
%\item \underline{曲線$C: \vec{p}(t) (a \leqq t \leqq b)$が区分的に滑らかな曲線}とは滑らかな曲線を端点でつないだもの.\end{itemize}

滑らかな曲線$C$に関してその長さを
$$
l(C) = \int_{a}^{b} \sqrt{(x'(t))^2 + (y'(t))^2} dt \text{\,\,\, とする.}
$$
     \end{dfn}
 \end{tcolorbox}
 
  \begin{tcolorbox}[
    colback = white,
    colframe = green!35!black,
    fonttitle = \bfseries,
    breakable = true]
    \begin{thm}
$f(x)$を$[a,b]$上の$C^1$級関数とする. このとき$y=f(x)$のグラフ$C=\{ (x, f(x))\,\,| \,\, a \leqq x \leqq b\}$の長さは
$$
l(C) = \int_{a}^{b} \sqrt{1 + (f'(x))^2} dx \text{\,\,\, である.}
$$
     \end{thm}
 \end{tcolorbox}
 
 \begin{exa}
 放物線$y=x^2 (0 \leqq x \leqq 1)$のグラフの長さを求めよ.
 
\hspace{-18pt} (答.) $f(x) = x^2$とすると$f'(x) = 2x$のため, 曲線の長さは\begin{align*}
\begin{split}
\int_{0}^{1} \sqrt{1 + (2x)^2} dx &= 
\int_{0}^{1} \sqrt{1 +4x^2} dx = 
\frac{1}{2}\int_{0}^{2} \sqrt{1 +t^2} dt \\
&= \frac{1}{4}\left[ t\sqrt{t^2 + 1} + \log \Bigl| t + \sqrt{t^2 + 1 }\Bigr| \right]^{2}_{0} \\
&= \frac{1}{4} \left(2 \sqrt{5} + \log (2 + \sqrt{5})\right)
\end{split}
\end{align*}
 \end{exa}

  \begin{tcolorbox}[
    colback = white,
    colframe = green!35!black,
    fonttitle = \bfseries,
    breakable = true]
    \begin{thm}
$[\alpha,\beta]$上の$C^1$級関数$f( \theta )$を用いて, 曲線$C$が
$$
\begin{array}{ccccc}
C: &[\alpha, \beta] & \rightarrow & \R^2 & \\
&\theta & \longmapsto &(f(\theta) \cos \theta, f(\theta) \sin \theta)&
\end{array}
$$
%$$C: (x(\theta), y(\theta)) = (f(\theta) \cos \theta, f(\theta) \sin \theta)$$
と表されているとき, $C$の長さは
$$
 \int_{\alpha}^{\beta} \sqrt{(f(\theta))^2 + (f'(\theta))^2} d\theta \text{\,\,\, である.}
$$
     \end{thm}
 \end{tcolorbox}
 
 \begin{exa}
(アルキメデスの螺旋)
正の実数$a , \alpha $について
$$
\begin{array}{ccccc}
C: &[0,\alpha] & \rightarrow & \R^2 & \\
&\theta & \longmapsto &(a \theta \cos \theta ,  a \theta \sin \theta)&
\end{array}
$$
 とする. 曲線$C$の長さを求めよ.
 
\hspace{-18pt} (答.) $f(\theta) = a \theta$とすると$f'(\theta) = a$のため, 曲線の長さは\begin{align*}
\begin{split}
\int_{0}^{\alpha} \sqrt{a^2 + (a \theta)^2} d\theta = 
a \int_{0}^{\alpha} \sqrt{1+ (\theta)^2} d\theta = 
\frac{a}{2}\left(\alpha\sqrt{\alpha^2 + 1} + \log (\alpha + \sqrt{\alpha^2 + 1})\right)
\end{split}
\end{align*}
 \end{exa}
 
\section{演習問題}
演習問題の解答は授業の黒板にあります.
\begin{enumerate}
\item 正の実数$a,b$について$f(x) = a \cosh \frac{x}{a} -a$とする.
グラフ$C=\{ (x, f(x))\,\,| \,\, 0 \leqq x \leqq b\}$の長さを$a,b$を用いて表せ.
\end{enumerate}



 

\end{document}
