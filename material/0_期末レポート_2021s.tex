\documentclass[dvipdfmx,a4paper,11pt]{article}
\usepackage[utf8]{inputenc}
%\usepackage[dvipdfmx]{hyperref} %リンクを有効にする
\usepackage{url} %同上
\usepackage{amsmath,amssymb} %もちろん
\usepackage{amsfonts,amsthm,mathtools} %もちろん
\usepackage{braket,physics} %あると便利なやつ
\usepackage{bm} %ラプラシアンで使った
\usepackage[top=30truemm,bottom=30truemm,left=25truemm,right=25truemm]{geometry} %余白設定
\usepackage{latexsym} %ごくたまに必要になる
\renewcommand{\kanjifamilydefault}{\gtdefault}
\usepackage{otf} %宗教上の理由でmin10が嫌いなので


\usepackage[all]{xy}
\usepackage{amsthm,amsmath,amssymb,comment}
\usepackage{amsmath}    % \UTF{00E6}\UTF{0095}°\UTF{00E5}\UTF{00AD}\UTF{00A6}\UTF{00E7}\UTF{0094}¨
\usepackage{amssymb}  
\usepackage{color}
\usepackage{amscd}
\usepackage{amsthm}  
\usepackage{wrapfig}
\usepackage{comment}	
\usepackage{graphicx}
\usepackage{setspace}
\setstretch{1.2}


\newcommand{\R}{\mathbb{R}}
\newcommand{\Z}{\mathbb{Z}}
\newcommand{\Q}{\mathbb{Q}} 
\newcommand{\N}{\mathbb{N}}
\newcommand{\C}{\mathbb{C}} 
\newcommand{\Sin}{\text{Sin}^{-1}} 
\newcommand{\Cos}{\text{Cos}^{-1}} 
\newcommand{\Tan}{\text{Tan}^{-1}} 
\newcommand{\invsin}{\text{Sin}^{-1}} 
\newcommand{\invcos}{\text{Cos}^{-1}} 
\newcommand{\invtan}{\text{Tan}^{-1}} 
\newcommand{\Area}{\text{Area}}
\newcommand{\vol}{\text{Vol}}




   %当然のようにやる.
\allowdisplaybreaks[4]
   %もちろん.
%\title{第1回. 多変数の連続写像 (岩井雅崇, 2020/10/06)}
%\author{岩井雅崇}
%\date{2020/10/06}
%ここまで今回の記事関係ない
\usepackage{tcolorbox}
\tcbuselibrary{breakable, skins, theorems}

\theoremstyle{definition}
\newtheorem{thm}{定理}
\newtheorem{lem}[thm]{補題}
\newtheorem{prop}[thm]{命題}
\newtheorem{cor}[thm]{系}
\newtheorem{claim}[thm]{主張}
\newtheorem{dfn}[thm]{定義}
\newtheorem{rem}[thm]{注意}
\newtheorem{exa}[thm]{例}
\newtheorem{conj}[thm]{予想}
\newtheorem{prob}[thm]{問題}
\newtheorem{rema}[thm]{補足}

\DeclareMathOperator{\Ric}{Ric}
\DeclareMathOperator{\Vol}{Vol}
 \newcommand{\pdrv}[2]{\frac{\partial #1}{\partial #2}}
 \newcommand{\drv}[2]{\frac{d #1}{d#2}}
  \newcommand{\ppdrv}[3]{\frac{\partial #1}{\partial #2 \partial #3}}



%ここから本文.
\begin{document}
%\maketitle
\begin{center}
{ \large 大阪市立大学 R3年度(2021年度)前期 全学共通科目 解析 I *TI電(都1\UTF{FF5E}28)} \\
\vspace{5pt}


{\LARGE 期末レポート } \\
\vspace{5pt}

{ \Large 提出締め切り 2021年7月27日 23時59分00秒 (日本標準時刻)}
\end{center}

\begin{flushright}
 担当教官: 岩井雅崇(いわいまさたか) 
\end{flushright}

{\Large $\bullet$ 注意事項}
\begin{enumerate}
\item 第1問から第4問まで解くこと. 
\item おまけ問題は全員が解く必要はない.(詳しくは成績の付け方のスライドを参照せよ).
\item 用語に関しては授業または教科書(三宅敏恒著 入門微分積分(培風館))に準じます.
\item \underline{提出締め切りを遅れて提出した場合, 大幅に減点する可能性がある.}
\item \underline{名前・学籍番号をきちんと書くこと.}
\item \underline{解答に関して, 答えのみならず, 答えを導出する過程をきちんと記してください.} きちんと記していない場合は大幅に減点する場合がある.
%ただし用語の定義の違いによるミスに関して, 大幅に減点することはない.
\item 字は汚くても構いませんが, \underline{読める字で濃く書いてください.} あまりにも読めない場合は採点をしないかもしれません.%\footnote{私も字が汚い方ですので人のこと言えませんが...自論ですが, 字が汚いと自覚ある人は大きく書けば読みやすくなると思います.}
\item 採点を効率的に行うため, \underline{順番通り解答するようお願いいたします.}
\item 採点を効率的に行うため,  \underline{レポートはpdfファイル形式で提出し,} ファイル名を「int(学籍番号).pdf」とするようお願いいたします. 
(intは積分(integral)の略です.)
例えば学籍番号が「A18CA999」の場合はファイル名は「intA18CA999.pdf」となります.
\end{enumerate}

 \begin{tcolorbox}[
    colback = white,
    colframe = black,
    fonttitle = \bfseries,
    breakable = true]
    レポート提出前のチェックリスト
    \begin{itemize}
    \item[] $\Box$ 締め切りを守っているか?
    \item[] $\Box$ レポートに名前・学籍番号を書いたか?
     \item[] $\Box$ 答えを導出する過程をきちんと記したか?
     \item[] $\Box$ 計算ミスしていないか?
    \item[] $\Box$ 他者が読める字で書いたか?
    \item[] $\Box$ 順番通り解答したか?
    \item[] $\Box$レポートはpdfファイル形式で提出したか?
   \item[] $\Box$ ファイル名を「int(学籍番号).pdf」としたか?
    \end{itemize}

  \end{tcolorbox}
  
%2020年12月15日(火)の10時50分からオンラインによる質疑応答の場を設けます. (出席義務はありません, 来たい人だけ来てください. レポートに関する質問も可とします.) 質疑応答に関してはWebClassを参照してください.
 
\newpage
 \hspace{-11pt}
{\Large $\bullet$ レポートの提出方法について }
\vspace{11pt}

\underline{原則的にWebClassからの提出しか認めません.}
レポートは余裕を持って提出してください.
\vspace{11pt}

\underline{レポートはpdfファイルで提出してください.}
またWebClassからの提出の際, 提出ファイルを一つにまとめる必要があるとのことですので, 提出ファイルを一つにまとめてください.
\vspace{11pt}

\underline{採点を効率的に行うため, ファイル名を「int(学籍番号).pdf」とするようお願いいたします.}
(intは積分(integral)の略です.)
例えば学籍番号が「A18CA999」の場合はファイル名は「intA18CA999.pdf」となります.

\vspace{11pt}
 \hspace{-11pt}
{\Large $\bullet$ 提出用pdfファイルの作成の仕方について}
\vspace{11pt}

いろいろ方法はあると思います.
\vspace{11pt}

1つ目は「手書きレポートをpdfにする方法」があります.
この方法は時間はあまりかかりませんが, お金がかかる可能性があります.
手書きレポートをpdfにするには以下の方法があると思います.
\begin{itemize}
\setlength{\parskip}{0cm} % 段落間
  \setlength{\itemsep}{0cm}
\item スキャナーを使うかコンビニに行ってスキャンする.
\item スマートフォンやカメラで画像データにしてからpdfにする. 例えばMicrosoft Wordを使えば画像データをpdfにできます. %(見づらくなる可能性あり)
\item その他いろいろ検索して独自の方法を行う.
\end{itemize}

2つ目は「TeXでレポートを作成する方法」があります.
時間はかなりかかりますが, 見た目はかなり綺麗です.
\vspace{11pt}


いずれの方法でも構いません. 最終的に私が読めるように書いたレポートであれば大丈夫です.
%他者が読める字で書いてあれば問題ありません. (私が読めるようなレポートであれば大丈夫です.)

\vspace{11pt}
 \hspace{-11pt}
{\Large $\bullet$ WebClassからの提出が不可能な場合}
\vspace{11pt}

提出の期限までに (WebClassのシステムトラブル等の理由で) WebClassからの提出が不可能な場合のみメール提出を受け付けます.
その場合には以下の項目を厳守してください.
\begin{itemize}
\setlength{\parskip}{0cm} % 段落間
  \setlength{\itemsep}{0cm}
\item 大学のメールアドレスを使って送信すること. (なりすまし提出防止のため.)
\item 件名を「レポート提出」とすること
\item 講義名, 学籍番号, 氏名 (フルネーム)を書くこと.
\item レポートのファイルを添付すること.
\item WebClassでの提出ができなかった事情を説明すること. (提出理由が不十分である場合, 減点となる可能性があります.)
\end{itemize}

メール提出の場合はmasataka[at]sci.osaka-cu.ac.jpにメールするようお願いいたします.

\newpage
\begin{center}
{\LARGE 期末レポート問題.} 
\end{center}

{\Large 第1問.} (授業第9,10回の内容.)
\vspace{11pt}

次の(1)から(4)までの不定積分を求めよ.


\vspace{11pt}

(1). $\int x e^{x^2} dx$\,\,\,
(2). $\int \frac{x^3 -1}{x^2 +1} dx$ \,\,\,
(3). $\int \frac{2}{x(x-1)(x-2)} dx$ \,\,\,
(4). $\int  \frac{1}{x \sqrt{x+1}} dx$
%\begin{enumerate}\item[問1.] $\sqrt{x}$\item[問2.] $\Tan x$\item[問3.] $\cosh x$\item[問4.] $\frac{x}{1-e^{-x}}$\end{enumerate}

\vspace{11pt}

\underline{ただし答えを導出する過程を記した上で, 答えは次のように書くこと.}

\vspace{11pt}

(例題) $ \int \sin x dx$ 

(答え) $\int \sin x dx = - \cos x$




 \vspace{33pt}
 
 {\Large 第2問.} (授業第9,10回の内容.)
 \vspace{11pt}
 
 
%(1). 次の定積分を求めよ.
%$$\int_{2}^{4} \frac{\sqrt{x}}{ \sqrt{6-x} + \sqrt{x}}dx$$
%(ヒント. $t=6-x$という置換積分法を用いる.)

(1). $a$を正の実数とし, $f(x)$を$[-a,a]$上の連続関数とする. 任意の$x \in [-a, a]$について$f(-x) = - f(x)$であると仮定する.このとき$\int_{-a}^{a} f(x) dx =0$であることを示せ.

    \vspace{11pt}
    
(2). 次の定積分を求めよ.

$$
\int_{-2}^{2} \left( x^3 \cos \frac{x}{2} + \frac{1}{2} \right) \sqrt{4 - x^2} dx
$$

%(ヒント. $t=-x$という置換積分法を用いる.)
   \vspace{33pt}
   
   {\Large 第3問.} (授業第9,10,11回の内容.)
    \vspace{11pt}
 
$$
I = \int_{0}^{\frac{\pi}{2}} \log (\sin x) dx
$$
とおく.
以下の問いに答えよ.
ただしこの広義積分が収束することは仮定してよい.
(したがって置換積分法や部分積分法などは自由に使って良い.)
     
     \vspace{11pt}
     
(1).  $I = \int_{0}^{\frac{\pi}{2}} \log (\cos x) dx$を示せ.

\vspace{11pt}

(2).  $2I = - \frac{\pi}{2}  \log 2 + \int_{0}^{\frac{\pi}{2}} \log (\sin 2x) dx$を示せ.

\vspace{11pt}

(3).  $\int_{0}^{\frac{\pi}{2}} \log (\sin 2x) dx$を$I$を用いて表せ.

\vspace{11pt}

(4). $I$の値を求めよ.

   %  \vspace{33pt} 
     
  \begin{flushright}
 {\LARGE 第4問に続く.}
 \end{flushright}
     
 \newpage
   
{\Large 第4問.} (授業第11回の内容.)
\vspace{11pt}

$p$を実数とする.
 広義積分
$$\int_{1}^{\infty} (1+2\sqrt{x})^{2p} \log x \text{\,}dx $$

が収束するような$p$の範囲を求めよ.
 
%$p$を実数とする.
%\vspace{11pt}{\large(1). $p<-1$のとき, 広義積分$\int_{1}^{\infty} (1+\sqrt{x})^{2p} \log x \text{\,}dx$が収束することを示せ.}

%\vspace{11pt}{\large(2). $p \geqq -1$のとき, 広義積分$\int_{1}^{\infty} (1+\sqrt{x})^{2p} \log x \text{\,}dx$が発散することを示せ.}

 \vspace{33pt}
 

{\Large 期末試験おまけ問題.} (授業第11回の内容.)
\vspace{11pt}

次の問いに答えよ.

\vspace{11pt}

(1). 広義積分
$$ \int_{0}^{\frac{1}{2}} \frac{1}{x (\log x)} dx$$
は収束するか発散するか. 理由とともに答えよ.
 
(2). 広義積分
$$ \int_{0}^{\frac{1}{2}} \frac{1}{x (\log x)^2} dx$$
は収束するか発散するか. 理由とともに答えよ.
 

     \vspace{33pt} 
     
 \begin{flushright}
 {\LARGE 以上.}
 \end{flushright}


%$f(x,y)$を$C^1$級関数とし,$C^1$級変数変換を$(x(u,v),y(u,v)) = (u \cos v, u \sin v)$とする.$g(u,v) = f(x(u,v), y(u,v))$とする時, $\pdrv{g}{u}, \pdrv{g}{v}$を$\pdrv{f}{x},\pdrv{f}{y}$を用いてあらわせ.

%$f(x,y) = x^3 -y^3 -3x +12y$について極大点・極小点を持つ点があれば, その座標と極値を求めよ. またその極値が極小値か極大値のどちらであるか示せ.

 %$f(x,y,z)= xyz, g(x,y,z)=x+y+z-170$とする.$g(x,y,z)=0$のもとで$f$の最大値を求めよ.つまり$S : = \{ (x,y) \in \R^3: g(x,y,z)=0\}$とする時, $f$の$S$上での最大値を求めよ.ただし$f$が$S$上で最大値を持つことは認めて良い.
  
 %

 

\end{document}
