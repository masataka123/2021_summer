\documentclass[dvipdfmx,a4paper,11pt]{article}
\usepackage[utf8]{inputenc}
%\usepackage[dvipdfmx]{hyperref} %リンクを有効にする
\usepackage{url} %同上
\usepackage{amsmath,amssymb} %もちろん
\usepackage{amsfonts,amsthm,mathtools} %もちろん
\usepackage{braket,physics} %あると便利なやつ
\usepackage{bm} %ラプラシアンで使った
\usepackage[top=30truemm,bottom=30truemm,left=25truemm,right=25truemm]{geometry} %余白設定
\usepackage{latexsym} %ごくたまに必要になる
\renewcommand{\kanjifamilydefault}{\gtdefault}
\usepackage{otf} %宗教上の理由でmin10が嫌いなので


\usepackage[all]{xy}
\usepackage{amsthm,amsmath,amssymb,comment}
\usepackage{amsmath}    % \UTF{00E6}\UTF{0095}°\UTF{00E5}\UTF{00AD}\UTF{00A6}\UTF{00E7}\UTF{0094}¨
\usepackage{amssymb}  
\usepackage{color}
\usepackage{amscd}
\usepackage{amsthm}  
\usepackage{wrapfig}
\usepackage{comment}	
\usepackage{graphicx}
\usepackage{setspace}
\setstretch{1.2}


\newcommand{\R}{\mathbb{R}}
\newcommand{\Z}{\mathbb{Z}}
\newcommand{\Q}{\mathbb{Q}} 
\newcommand{\N}{\mathbb{N}}
\newcommand{\C}{\mathbb{C}} 
\newcommand{\Sin}{\text{Sin}^{-1}} 
\newcommand{\Cos}{\text{Cos}^{-1}} 
\newcommand{\Tan}{\text{Tan}^{-1}} 
\newcommand{\invsin}{\text{Sin}^{-1}} 
\newcommand{\invcos}{\text{Cos}^{-1}} 
\newcommand{\invtan}{\text{Tan}^{-1}} 
\newcommand{\Area}{\text{Area}}
\newcommand{\vol}{\text{Vol}}




   %当然のようにやる.
\allowdisplaybreaks[4]
   %もちろん.
%\title{第1回. 多変数の連続写像 (岩井雅崇, 2020/10/06)}
%\author{岩井雅崇}
%\date{2020/10/06}
%ここまで今回の記事関係ない
\usepackage{tcolorbox}
\tcbuselibrary{breakable, skins, theorems}

\theoremstyle{definition}
\newtheorem{thm}{定理}
\newtheorem{lem}[thm]{補題}
\newtheorem{prop}[thm]{命題}
\newtheorem{cor}[thm]{系}
\newtheorem{claim}[thm]{主張}
\newtheorem{dfn}[thm]{定義}
\newtheorem{rem}[thm]{注意}
\newtheorem{exa}[thm]{例}
\newtheorem{conj}[thm]{予想}
\newtheorem{prob}[thm]{問題}
\newtheorem{rema}[thm]{補足}

\DeclareMathOperator{\Ric}{Ric}
\DeclareMathOperator{\Vol}{Vol}
 \newcommand{\pdrv}[2]{\frac{\partial #1}{\partial #2}}
 \newcommand{\drv}[2]{\frac{d #1}{d#2}}
  \newcommand{\ppdrv}[3]{\frac{\partial #1}{\partial #2 \partial #3}}


%ここから本文.
\begin{document}
%\maketitle
\begin{center}
{\Large 第4回. 平均値の定理と関数の極限値計算 (三宅先生の本, 2.2の内容)}
\end{center}

\begin{flushright}
 岩井雅崇 2021/05/11
\end{flushright}

\section{関数の極値}
%以下この授業を通してよく使う記号や用語をまとめる.(興味がなければ飛ばして良い)

%\subsection{関数}
\begin{tcolorbox}[
    colback = white,
    colframe = green!35!black,
    fonttitle = \bfseries,
    breakable = true]
    \begin{dfn}[極値]
$f(x)$を区間$I$上の関数とする.
\begin{itemize}
\item \underline{$f(x)$が$c\in I$で極大}であるとは, $c$を含む開区間$J$があって, $x \in J$かつ$x \neq c$ならば$f(x) < f(c)$となること.
このとき, \underline{$f(x)$は$c$で極大である}といい, $f(c)$の値を\underline{極大値}という.
\item \underline{$f(x)$が$c\in I$で極小}であるとは, $c$を含む開区間$J$があって, $x \in J$かつ$x \neq c$ならば$f(x) > f(c)$となること.
このとき, \underline{$f(x)$は$c$で極小である}といい, $f(c)$の値を\underline{極小値}という.
\item  極大値, 極小値の二つ合わせて\underline{極値}という.%極値をとる点$(a,b)$を\underline{極値点}という.
\end{itemize}

    \end{dfn}
\end{tcolorbox}

\begin{tcolorbox}[
    colback = white,
    colframe = green!35!black,
    fonttitle = \bfseries,
    breakable = true]
    \begin{thm}
    $f(x)$を$[a,b]$上で連続, $(a,b)$上で微分可能な関数とする.
    $f(x)$が$c \in (a,b)$で極値を持てば, $f'(c) = 0$である.
    \end{thm}
\end{tcolorbox}

\section{平均値の定理とその応用}
\begin{tcolorbox}[
    colback = white,
    colframe = green!35!black,
    fonttitle = \bfseries,
    breakable = true]
    \begin{thm}
    $f(x), g(x)$を$[a,b]$上で連続, $(a,b)$上で微分可能な関数とする.
\begin{itemize}
\item (ロルの定理) $f(a) = f(b)$ならば, $f'(c) = 0$となる$c \in (a,b)$がある.
\item (平均値の定理)
$$
f'(c) = \frac{f(b)-f(a)}{b-a}
$$
となる$c \in (a,b)$が存在する. 
\item (コーシーの平均値の定理)
$g(a) \neq g(b)$かつ任意の$x \in (a,b)$について$g'(x) \neq 0$ならば
$$
\frac{f'(c)}{g'(c)} = \frac{f(b)-f(a)}{g(b)-g(a)}
$$
となる$c \in (a,b)$が存在する. 
\end{itemize}

    \end{thm}
\end{tcolorbox}

\begin{tcolorbox}[
    colback = white,
    colframe = green!35!black,
    fonttitle = \bfseries,
    breakable = true]
    \begin{thm}
    $f(x)$を$[a,b]$上で連続, $(a,b)$上で微分可能な関数とする.
\begin{itemize}
\item 任意の$x \in (a,b)$について$f'(x)=0$ならば$f$は$[a,b]$上で定数関数.
\item 任意の$x \in (a,b)$について$f'(x)>0$ならば$f$は$[a,b]$上で単調増加関数.
\end{itemize}

    \end{thm}
\end{tcolorbox}

\begin{exa}
$(\sin x)' = \cos x$より, $\sin x$は$[-\frac{\pi}{2}, \frac{\pi}{2}]$上単調増加.
\end{exa}

\begin{tcolorbox}[
    colback = white,
    colframe = green!35!black,
    fonttitle = \bfseries,
    breakable = true]
    \begin{thm}[ロピタルの定理]
    $f(x), g(x)$を点$a$の近くで定義された微分可能な関数とする.
$\lim_{x \rightarrow a} f(x) = \lim_{x \rightarrow a} g(x) =0$かつ
$\lim_{x \rightarrow a} \frac{f'(x)}{g'(x)}$が存在するならば, 
$\lim_{x \rightarrow a} \frac{f(x)}{g(x)}$も存在して
$$
\lim_{x \rightarrow a} \frac{f(x)}{g(x)} = \lim_{x \rightarrow a} \frac{f'(x)}{g'(x)}.$$
    \end{thm}
\end{tcolorbox}
\begin{exa}
$$
\lim_{x \rightarrow 0} \frac{e^{2x} - \cos x}{x} \text{\,\,を求めよ.}
$$
(答.)
$\lim_{x \rightarrow 0} e^{2x} - \cos x =1-1=0$かつ
$\lim_{x \rightarrow 0} x=0$であり
$$
\lim_{x \rightarrow 0} \frac{(e^{2x} - \cos x)'}{(x)'}
=
\lim_{x \rightarrow 0} \frac{2 e^{2x} - \sin x}{1} =2
$$
であるため, ロピタルの定理から
$$
\lim_{x \rightarrow 0} \frac{e^{2x} - \cos x}{x} =
\lim_{x \rightarrow 0} \frac{(e^{2x} - \cos x)'}{(x)'}
=2
$$
\end{exa}





\section{演習問題}
演習問題の解答は授業の黒板にあります.
\begin{enumerate}
\item 
$$
\lim_{x \rightarrow 0} \frac{x - \sin x}{x^3} \text{\,\,を求めよ.}
$$
\end{enumerate}



 

\end{document}
