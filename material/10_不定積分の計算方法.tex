\documentclass[dvipdfmx,a4paper,11pt]{article}
\usepackage[utf8]{inputenc}
%\usepackage[dvipdfmx]{hyperref} %リンクを有効にする
\usepackage{url} %同上
\usepackage{amsmath,amssymb} %もちろん
\usepackage{amsfonts,amsthm,mathtools} %もちろん
\usepackage{braket,physics} %あると便利なやつ
\usepackage{bm} %ラプラシアンで使った
\usepackage[top=30truemm,bottom=30truemm,left=25truemm,right=25truemm]{geometry} %余白設定
\usepackage{latexsym} %ごくたまに必要になる
\renewcommand{\kanjifamilydefault}{\gtdefault}
\usepackage{otf} %宗教上の理由でmin10が嫌いなので


\usepackage[all]{xy}
\usepackage{amsthm,amsmath,amssymb,comment}
\usepackage{amsmath}    % \UTF{00E6}\UTF{0095}°\UTF{00E5}\UTF{00AD}\UTF{00A6}\UTF{00E7}\UTF{0094}¨
\usepackage{amssymb}  
\usepackage{color}
\usepackage{amscd}
\usepackage{amsthm}  
\usepackage{wrapfig}
\usepackage{comment}	
\usepackage{graphicx}
\usepackage{setspace}
\setstretch{1.2}


\newcommand{\R}{\mathbb{R}}
\newcommand{\Z}{\mathbb{Z}}
\newcommand{\Q}{\mathbb{Q}} 
\newcommand{\N}{\mathbb{N}}
\newcommand{\C}{\mathbb{C}} 
\newcommand{\Sin}{\text{Sin}^{-1}} 
\newcommand{\Cos}{\text{Cos}^{-1}} 
\newcommand{\Tan}{\text{Tan}^{-1}} 
\newcommand{\invsin}{\text{Sin}^{-1}} 
\newcommand{\invcos}{\text{Cos}^{-1}} 
\newcommand{\invtan}{\text{Tan}^{-1}} 
\newcommand{\Area}{\text{Area}}
\newcommand{\vol}{\text{Vol}}




   %当然のようにやる.
\allowdisplaybreaks[4]
   %もちろん.
%\title{第1回. 多変数の連続写像 (岩井雅崇, 2020/10/06)}
%\author{岩井雅崇}
%\date{2020/10/06}
%ここまで今回の記事関係ない
\usepackage{tcolorbox}
\tcbuselibrary{breakable, skins, theorems}

\theoremstyle{definition}
\newtheorem{thm}{定理}
\newtheorem{lem}[thm]{補題}
\newtheorem{prop}[thm]{命題}
\newtheorem{cor}[thm]{系}
\newtheorem{claim}[thm]{主張}
\newtheorem{dfn}[thm]{定義}
\newtheorem{rem}[thm]{注意}
\newtheorem{exa}[thm]{例}
\newtheorem{conj}[thm]{予想}
\newtheorem{prob}[thm]{問題}
\newtheorem{rema}[thm]{補足}

\DeclareMathOperator{\Ric}{Ric}
\DeclareMathOperator{\Vol}{Vol}
 \newcommand{\pdrv}[2]{\frac{\partial #1}{\partial #2}}
 \newcommand{\drv}[2]{\frac{d #1}{d#2}}
  \newcommand{\ppdrv}[3]{\frac{\partial #1}{\partial #2 \partial #3}}


%ここから本文.
\begin{document}
%\maketitle
\begin{center}
{\Large 第10回. 不定積分の計算方法 (三宅先生の本, 3.2の内容)}
\end{center}

\begin{flushright}
 岩井雅崇 2021/06/22
\end{flushright}


\section{不定積分の計算方法・テクニック}
\subsection{有理式の場合}
 
\begin{tcolorbox}[
    colback = white,
    colframe = green!35!black,
    fonttitle = \bfseries,
    breakable = true]
    \begin{dfn}[有理式]
  $f(x)$と$g(x)$を実数係数多項式とするとき, \underline{$\frac{f(x)}{g(x)}$を有理式}という.
        \end{dfn}
    \end{tcolorbox}

以下$f(x)$と$g(x)$を同時に割り切る多項式はないものと仮定する.(つまり$f(x)$と$g(x)$は互いに素とする.)

\begin{tcolorbox}[
    colback = white,
    colframe = green!35!black,
    fonttitle = \bfseries,
    breakable = true]
    \begin{thm}[有理式]
有理式$\frac{f(x)}{g(x)}$は次の3つの式の和に分解できる.
\begin{enumerate}
\item 多項式
\item $\frac{a}{(x+b)^m}$ ($a,b \in \R, m\in \N$)
\item $\frac{ax + b}{(x^2 + cx +d)^m}$ ($a,b,c,d \in \R, m\in \N$)
\end{enumerate}
特に$\alpha_1, \ldots, \alpha_l \in \R$と$m_1, \ldots, m_l \in \N$を用いて
$g(x) = (x- \alpha_1)^{m_1} \cdots (x- \alpha_l)^{m_l} $と書けるとき, 
有理式$\frac{f(x)}{g(x)}$は多項式と$\frac{\beta_i}{(x- \alpha_i)^m}$ ($\beta_i \in \R, m\in \N$, $1 \leqq m\leqq m_i$)の和で表せられる.
        \end{thm}
    \end{tcolorbox}
\begin{exa}
$\frac{5x -4 }{2x^2 + x -6}$に関して上の定理より, 
$$
\frac{5x -4 }{2x^2 + x -6} = \frac{a}{2x-3} + \frac{b}{x+2}
$$
となる実数$a,b \in \R$が存在する.
通分して計算すると$a=1, b=2$をえる.

\end{exa}
\begin{exa}
$$
\frac{2}{(x-1)(x^2 + 1)} = \frac{1}{x-1} - \frac{x+1}{x^2 + 1}
$$
\end{exa}

\begin{tcolorbox}[
    colback = white,
    colframe = green!35!black,
    fonttitle = \bfseries,
    breakable = true]
    \begin{thm}
有理式の不定積分は計算できる.
        \end{thm}
    \end{tcolorbox}
\begin{exa}
$$
\int \frac{5x -4 }{2x^2 + x -6}dx = \int \frac{1}{2x-3} dx+ \int \frac{2}{x+2}dx
= \frac{1}{2}\log|2x-3| + 2 \log |x+2|
$$
$$
\int \frac{2}{(x-1)(x^2 + 1)} dx = \int \frac{1}{x-1}dx  - \int \frac{x+1}{x^2 + 1}dx
= \log|x-1| - \frac{1}{2} \log |x^2+1|- \Tan x
$$
\end{exa}

\subsection{無理関数がある場合}
テクニックだけまとめておく.
\begin{itemize}
\item $\sqrt[n]{ax + b}$に関して, $t= \sqrt[n]{ax + b}$とおくと
$$
x = \frac{t^n - b}{a}, dx = \frac{n t^{n-1} dt}{a}
$$
より有理式に帰着できる.
\item $\sqrt{ax^2 + b x + c}$に関して, $a>0$ならば$\sqrt{ax^2 + b x + c} = t - \sqrt{a} x$とおく.
\item $\sqrt{ax^2 + b x + c}$に関して, $ax^2 + b x + c = (x - \alpha)(x- \beta)$となる実数$\alpha, \beta \in \R$があるとき, $t = \sqrt{\frac{a(x - \beta)}{(x-\alpha)}}$とおく.
\end{itemize}

\begin{exa}
不定積分$\int \frac{dx}{x + 2 \sqrt{x-1}}$を求めよ.

\hspace{-18pt}(答.) $t = \sqrt{x-1}$とおくと, $x = t^2 +1, dx = 2tdt$より
\begin{align*}
\begin{split}
\int \frac{dx}{x + 2 \sqrt{x-1}}
&= \int \frac{2t}{t^2 + 1 + 2t} dt= \int \frac{2t + 2 -2}{(t + 1)^2} dt \\
&=  \int \frac{2}{t + 1} dt -  \int \frac{2}{(t + 1)^2} dt \\
&= 2 \log |t+1| + \frac{2}{t+1} \\
&= 2 \log (1 + \sqrt{x-1}) + \frac{2}{1 + \sqrt{x-1}}
\end{split}
\end{align*}

\end{exa}

\subsection{三角関数の有理式の積分}

\begin{tcolorbox}[
    colback = white,
    colframe = green!35!black,
    fonttitle = \bfseries,
    breakable = true]
    \begin{thm}
三角関数に関する有理式の不定積分は計算できる.
具体的には$t = \tan \frac{x}{2}$とおけば, $\sin x$などは次のように表される.
\begin{itemize}
\item $dx = \frac{2 dt}{1+ t^2}$
\item $\sin x = \frac{2t}{1+ t^2}$
\item $\cos x = \frac{1 - t^2}{1+ t^2}$
\item $\tan x = \frac{2t}{1- t^2}$
\end{itemize}


        \end{thm}
    \end{tcolorbox}
    
\begin{exa}
不定積分$\int \frac{1 + \sin x}{1 + \cos x} dx$を求めよ.

\hspace{-18pt}(答.) $t = \tan \frac{x}{2}$とおくと, 
\begin{align*}
\begin{split}
\int \frac{1 + \sin x}{1 + \cos x} dx
&= \int \frac{1 + \frac{2t}{1+t^2} }{ 1+ \frac{1-t^2}{1+ t^2}} 
\frac{2dt}{1+ t^2} = \int \frac{t^2 + 2t + 1}{1+ t^2} dt = \int 1 +\frac{2t}{1+ t^2} dt \\
&= t + \log (1 + t^2) = \tan \frac{x}{2} + \log \left|1 + \left(\tan \frac{x}{2} \right)^2 \right|\\
\end{split}
\end{align*}

\end{exa}


\section{演習問題}
演習問題の解答は授業の黒板にあります.
\begin{enumerate}
\item 不定積分$\int \frac{x^2}{x^2 - x - 6}dx$を求めよ.
\end{enumerate}



 

\end{document}
