\documentclass[dvipdfmx,a4paper,11pt]{article}
\usepackage[utf8]{inputenc}
%\usepackage[dvipdfmx]{hyperref} %リンクを有効にする
\usepackage{url} %同上
\usepackage{amsmath,amssymb} %もちろん
\usepackage{amsfonts,amsthm,mathtools} %もちろん
\usepackage{braket,physics} %あると便利なやつ
\usepackage{bm} %ラプラシアンで使った
\usepackage[top=30truemm,bottom=30truemm,left=25truemm,right=25truemm]{geometry} %余白設定
\usepackage{latexsym} %ごくたまに必要になる
\renewcommand{\kanjifamilydefault}{\gtdefault}
\usepackage{otf} %宗教上の理由でmin10が嫌いなので


\usepackage[all]{xy}
\usepackage{amsthm,amsmath,amssymb,comment}
\usepackage{amsmath}    % \UTF{00E6}\UTF{0095}°\UTF{00E5}\UTF{00AD}\UTF{00A6}\UTF{00E7}\UTF{0094}¨
\usepackage{amssymb}  
\usepackage{color}
\usepackage{amscd}
\usepackage{amsthm}  
\usepackage{wrapfig}
\usepackage{comment}	
\usepackage{graphicx}
\usepackage{setspace}
\setstretch{1.2}


\newcommand{\R}{\mathbb{R}}
\newcommand{\Z}{\mathbb{Z}}
\newcommand{\Q}{\mathbb{Q}} 
\newcommand{\N}{\mathbb{N}}
\newcommand{\C}{\mathbb{C}} 
\newcommand{\Sin}{\text{Sin}^{-1}} 
\newcommand{\Cos}{\text{Cos}^{-1}} 
\newcommand{\Tan}{\text{Tan}^{-1}} 
\newcommand{\invsin}{\text{Sin}^{-1}} 
\newcommand{\invcos}{\text{Cos}^{-1}} 
\newcommand{\invtan}{\text{Tan}^{-1}} 
\newcommand{\Area}{\text{Area}}
\newcommand{\vol}{\text{Vol}}




   %当然のようにやる.
\allowdisplaybreaks[4]
   %もちろん.
%\title{第1回. 多変数の連続写像 (岩井雅崇, 2020/10/06)}
%\author{岩井雅崇}
%\date{2020/10/06}
%ここまで今回の記事関係ない
\usepackage{tcolorbox}
\tcbuselibrary{breakable, skins, theorems}

\theoremstyle{definition}
\newtheorem{thm}{定理}
\newtheorem{lem}[thm]{補題}
\newtheorem{prop}[thm]{命題}
\newtheorem{cor}[thm]{系}
\newtheorem{claim}[thm]{主張}
\newtheorem{dfn}[thm]{定義}
\newtheorem{rem}[thm]{注意}
\newtheorem{exa}[thm]{例}
\newtheorem{conj}[thm]{予想}
\newtheorem{prob}[thm]{問題}
\newtheorem{rema}[thm]{補足}

\DeclareMathOperator{\Ric}{Ric}
\DeclareMathOperator{\Vol}{Vol}
 \newcommand{\pdrv}[2]{\frac{\partial #1}{\partial #2}}
 \newcommand{\drv}[2]{\frac{d #1}{d#2}}
  \newcommand{\ppdrv}[3]{\frac{\partial #1}{\partial #2 \partial #3}}



%ここから本文.
\begin{document}
%\maketitle
\begin{center}
{ \large 大阪市立大学 R3年度(2021年度)前期 全学共通科目 解析 I *TI電(都1\UTF{FF5E}28)} \\
\vspace{5pt}

{\LARGE 中間レポート } \\
\vspace{5pt}

{ \Large 提出締め切り 2021年6月8日 23時59分00秒 (日本標準時刻)}
\end{center}

\begin{flushright}
 担当教官: 岩井雅崇(いわいまさたか) 
\end{flushright}

{\Large $\bullet$ 注意事項}
\begin{enumerate}
\item 第1問から第4問まで解くこと. 
\item おまけ問題は全員が解く必要はない.(詳しくは成績の付け方のスライドを参照せよ).
\item 用語に関しては授業または教科書(三宅敏恒著 入門微分積分(培風館))に準じます.
\item \underline{提出締め切りを遅れて提出した場合, 大幅に減点する可能性がある.}
\item \underline{名前・学籍番号をきちんと書くこと.}
\item \underline{解答に関して, 答えのみならず, 答えを導出する過程をきちんと記してください.} きちんと記していない場合は大幅に減点する場合がある.
%ただし用語の定義の違いによるミスに関して, 大幅に減点することはない.
\item 字は汚くても構いませんが, \underline{読める字で濃く書いてください.} あまりにも読めない場合は採点をしないかもしれません.%\footnote{私も字が汚い方ですので人のこと言えませんが...自論ですが, 字が汚いと自覚ある人は大きく書けば読みやすくなると思います.}
\item 採点を効率的に行うため, \underline{順番通り解答するようお願いいたします.}
\item 採点を効率的に行うため,  \underline{レポートはpdfファイル形式で提出し,} ファイル名を「dif(学籍番号).pdf」とするようお願いいたします. 
(difは微分(differential)の略です.)
例えば学籍番号が「A18CA999」の場合はファイル名は「difA18CA999.pdf」となります.
\end{enumerate}

 \begin{tcolorbox}[
    colback = white,
    colframe = black,
    fonttitle = \bfseries,
    breakable = true]
    レポート提出前のチェックリスト
    \begin{itemize}
    \item[] $\Box$ 締め切りを守っているか?
    \item[] $\Box$ レポートに名前・学籍番号を書いたか?
     \item[] $\Box$ 答えを導出する過程をきちんと記したか?
     \item[] $\Box$ 計算ミスしていないか?
    \item[] $\Box$ 他者が読める字で書いたか?
    \item[] $\Box$ 順番通り解答したか?
    \item[] $\Box$レポートはpdfファイル形式で提出したか?
   \item[] $\Box$ ファイル名を「dif(学籍番号).pdf」としたか?
    \end{itemize}

  \end{tcolorbox}
  
%2020年12月15日(火)の10時50分からオンラインによる質疑応答の場を設けます. (出席義務はありません, 来たい人だけ来てください. レポートに関する質問も可とします.) 質疑応答に関してはWebClassを参照してください.
 
\newpage
 \hspace{-11pt}
{\Large $\bullet$ レポートの提出方法について }
\vspace{11pt}

\underline{原則的にWebClassからの提出しか認めません.}
レポートは余裕を持って提出してください.
\vspace{11pt}

\underline{レポートはpdfファイルで提出してください.}
またWebClassからの提出の際, 提出ファイルを一つにまとめる必要があるとのことですので, 提出ファイルを一つにまとめてください.
\vspace{11pt}

\underline{採点を効率的に行うため, ファイル名を「dif(学籍番号).pdf」とするようお願いいたします.}
(difは微分(differential)の略です.)
例えば学籍番号が「A18CA999」の場合はファイル名は「difA18CA999.pdf」となります.

\vspace{11pt}
 \hspace{-11pt}
{\Large $\bullet$ 提出用pdfファイルの作成の仕方について}
\vspace{11pt}

いろいろ方法はあると思います.
\vspace{11pt}

1つ目は「手書きレポートをpdfにする方法」があります.
この方法は時間はあまりかかりませんが, お金がかかる可能性があります.
手書きレポートをpdfにするには以下の方法があると思います.
\begin{itemize}
\setlength{\parskip}{0cm} % 段落間
  \setlength{\itemsep}{0cm}
\item スキャナーを使うかコンビニに行ってスキャンする.
\item スマートフォンやカメラで画像データにしてからpdfにする. 例えばMicrosoft Wordを使えば画像データをpdfにできます. %(見づらくなる可能性あり)
\item その他いろいろ検索して独自の方法を行う.
\end{itemize}

2つ目は「TeXでレポートを作成する方法」があります.
時間はかなりかかりますが, 見た目はかなり綺麗です.
\vspace{11pt}


いずれの方法でも構いません. 最終的に私が読めるように書いたレポートであれば大丈夫です.
%他者が読める字で書いてあれば問題ありません. (私が読めるようなレポートであれば大丈夫です.)

\vspace{11pt}
 \hspace{-11pt}
{\Large $\bullet$ WebClassからの提出が不可能な場合}
\vspace{11pt}

提出の期限までに (WebClassのシステムトラブル等の理由で) WebClassからの提出が不可能な場合のみメール提出を受け付けます.
その場合には以下の項目を厳守してください.
\begin{itemize}
\setlength{\parskip}{0cm} % 段落間
  \setlength{\itemsep}{0cm}
\item 大学のメールアドレスを使って送信すること. (なりすまし提出防止のため.)
\item 件名を「レポート提出」とすること
\item 講義名, 学籍番号, 氏名 (フルネーム)を書くこと.
\item レポートのファイルを添付すること.
\item WebClassでの提出ができなかった事情を説明すること. (提出理由が不十分である場合, 減点となる可能性があります.)
\end{itemize}

メール提出の場合はmasataka[at]sci.osaka-cu.ac.jpにメールするようお願いいたします.

%正当な理由(WebClassのシステムトラブル等)ではない場合, メールでの提出は減点対象となるので注意すること.
\newpage
\begin{center}
{\LARGE 中間レポート問題.} 
\end{center}

{\Large 第1問.} (授業第6回の内容.)
\vspace{11pt}

次の(1)から(4)までの関数について, $x=0$の近くでの漸近展開を3次まで求めよ.


\vspace{11pt}

(1). $\sqrt{x+1}$\,\,\,
(2). $\Tan x$ \,\,\,
(3). $\cosh x$ \,\,\,
(4). $(1+ x) \sin x$
%(4). $\frac{x}{1-e^{-x}}$
%\begin{enumerate}\item[問1.] $\sqrt{x}$\item[問2.] $\Tan x$\item[問3.] $\cosh x$\item[問4.] $\frac{x}{1-e^{-x}}$\end{enumerate}

\vspace{11pt}

\underline{ただし答えを導出する過程を記した上で, 答えは次のように書くこと.}

\vspace{11pt}

(例題1) $e^x$ 

(答え) $e^x = 1 + x + \frac{x^2}{2} + \frac{x^3}{6}  + o(x^3)$\,\,\,
$(x \rightarrow 0)$


\vspace{11pt}

(例題2) $\cos x$

(答え) $\cos x= 1 - \frac{x^2}{2} + o(x^3)$\,\,\,$(x \rightarrow 0)$



 \vspace{33pt}
 
 {\Large 第2問.} (授業第5回の内容.)
 \vspace{11pt}
 
(1).
$[0, \frac{\pi}{2}]$上で次の不等式が成り立つことを示せ.
$$
x - \frac{x^3}{3!} + \frac{x^5}{5!} - \frac{x^7}{7!}
\leqq \sin x
\leqq  
x - \frac{x^3}{3!} + \frac{x^5}{5!}
$$

\vspace{11pt}

(2).
$\sin 1$を小数第3位まで求めよ.
  
   \vspace{33pt}
   
   {\Large 第3問.} (授業第3回の内容.)
    \vspace{11pt}
 
 $\theta = \Tan \left( \frac{1}{5} \right)$とおく.
 次の問いに答えよ.
 
 \vspace{11pt}
 
(1).  $\tan(2 \theta)$の値を求めよ.

    \vspace{11pt}
    
(2).  $\tan(4 \theta)$の値を求めよ.

\vspace{11pt}

(3).  $\tan(4 \theta - \frac{\pi}{4})$の値を求めよ.

\vspace{11pt}

(4).  次の式を示せ.
$$
\frac{\pi}{4} = 4 \, \Tan \left(\frac{1}{5} \right)-  \Tan \left( \frac{1}{239} \right)
$$

     \vspace{33pt} 
     
  \begin{flushright}
 {\LARGE 第4問に続く.}
 \end{flushright}
     
 \newpage
   
{\Large 第4問.} (授業第4回の内容.)
    \vspace{11pt}

 $
 f(x) = \frac{x^3 -1 }{2}
 $
 とおく. 次の問いに答えよ.
 
 \vspace{11pt}
 
 (1). $f(x)$は$\R$上で単調増加であることを示せ.
 
 \vspace{11pt}
 
 (2). 方程式$ \sqrt[3]{2x + 1} = \frac{x^3 - 1}{2}$の実数解を全て求めよ.
 
\vspace{33pt} 

{\Large 中間レポートおまけ問題.} (授業第1,2,5回の内容.)
\vspace{11pt}

次の問いに答えよ.

\vspace{11pt}

(1).
地球上ではいかなる時間でも, ある赤道上の点$x$とその対蹠点$y$で, $x$と$y$の気温が同じような点の組$(x,y)$が存在することを示せ.
ただし地球は球体であると仮定して良い.
ここで対蹠点とは地球の裏側の点のことである.\footnote{例えば杉本キャンパスは東経135度北緯34度に位置しているため, 杉本キャンパスの対蹠点は西経45度南緯34度である. 調べてみるとおそらくパラグアイの領海か排他的経済水域のように思われる. ちなみに杉本キャンパスの裏側はブラジルではない.}

\vspace{11pt}
 
(2). $a<b$なる任意の実数$a,b$について, $a<q<b$となる有理数$q$が存在することを示せ.

\vspace{11pt}

(3). $\sqrt[3]{3}$の小数第10位までを求めるプログラムを作成せよ.
ただし次の点に注意すること.
\begin{enumerate}
\item[注意1.] プログラミング言語に関しては自由だが, あまりにもマニアックな言語は控えてください.\footnote{Haskellは大丈夫です. 私はc, c++, pythonぐらいなら読めます.}
\item[注意2.] \underline{この問題は$\sqrt[3]{3}$の値を求めるアルゴリズムを考える問題である.}
そのため\text{print(pow(3,$\frac{1}{3}$))}という解答などは不正解とする.
\item[注意3.] 処理時間があまりにも長い場合は不正解とする. 処理時間の目安は2秒とする.
\item[注意4.] 提出方法に関してはgithub等にアップロードしてそのリンクをレポートに貼っても良いし, プログラムのソースファイルを(スクリーンショット等で)画像にしてその画像をそのままレポートに貼っても良いです. (メールやWebClassのダイレクトメッセージで, プログラムのソースファイルを直接私に送るなどでも良いです.) \\
 \underline{この問題に限り, 提出方法は皆さんにお任せいたします. }
%目安として処理時間2秒程度とする.
\end{enumerate}

%任意の点$(a,b)\in \R^2$と任意の正の数$r>0$について, 点$(a,b)$中心の半径$r$の円板を$$B =\{ (x,y) \in \R^2 : \sqrt{(x-a)^2 + (y-b)^2} \leqq r\} \text{\,\, とするとき, }$$  ある点$(c,d) \in B$があって, $c$も$d$も共に有理数である.

     \vspace{33pt} 
     
 \begin{flushright}
 {\LARGE 以上.}
 \end{flushright}


%$f(x,y)$を$C^1$級関数とし,$C^1$級変数変換を$(x(u,v),y(u,v)) = (u \cos v, u \sin v)$とする.$g(u,v) = f(x(u,v), y(u,v))$とする時, $\pdrv{g}{u}, \pdrv{g}{v}$を$\pdrv{f}{x},\pdrv{f}{y}$を用いてあらわせ.

%$f(x,y) = x^3 -y^3 -3x +12y$について極大点・極小点を持つ点があれば, その座標と極値を求めよ. またその極値が極小値か極大値のどちらであるか示せ.

 %$f(x,y,z)= xyz, g(x,y,z)=x+y+z-170$とする.$g(x,y,z)=0$のもとで$f$の最大値を求めよ.つまり$S : = \{ (x,y) \in \R^3: g(x,y,z)=0\}$とする時, $f$の$S$上での最大値を求めよ.ただし$f$が$S$上で最大値を持つことは認めて良い.
  
 %

 

\end{document}
