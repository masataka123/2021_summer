\documentclass[dvipdfmx,a4paper,11pt]{article}
\usepackage[utf8]{inputenc}
%\usepackage[dvipdfmx]{hyperref} %リンクを有効にする
\usepackage{url} %同上
\usepackage{amsmath,amssymb} %もちろん
\usepackage{amsfonts,amsthm,mathtools} %もちろん
\usepackage{braket,physics} %あると便利なやつ
\usepackage{bm} %ラプラシアンで使った
\usepackage[top=30truemm,bottom=30truemm,left=25truemm,right=25truemm]{geometry} %余白設定
\usepackage{latexsym} %ごくたまに必要になる
\renewcommand{\kanjifamilydefault}{\gtdefault}
\usepackage{otf} %宗教上の理由でmin10が嫌いなので


\usepackage[all]{xy}
\usepackage{amsthm,amsmath,amssymb,comment}
\usepackage{amsmath}    % \UTF{00E6}\UTF{0095}°\UTF{00E5}\UTF{00AD}\UTF{00A6}\UTF{00E7}\UTF{0094}¨
\usepackage{amssymb}  
\usepackage{color}
\usepackage{amscd}
\usepackage{amsthm}  
\usepackage{wrapfig}
\usepackage{comment}	
\usepackage{graphicx}
\usepackage{setspace}
\setstretch{1.2}


\newcommand{\R}{\mathbb{R}}
\newcommand{\Z}{\mathbb{Z}}
\newcommand{\Q}{\mathbb{Q}} 
\newcommand{\N}{\mathbb{N}}
\newcommand{\C}{\mathbb{C}} 
\newcommand{\Sin}{\text{Sin}^{-1}} 
\newcommand{\Cos}{\text{Cos}^{-1}} 
\newcommand{\Tan}{\text{Tan}^{-1}} 
\newcommand{\invsin}{\text{Sin}^{-1}} 
\newcommand{\invcos}{\text{Cos}^{-1}} 
\newcommand{\invtan}{\text{Tan}^{-1}} 
\newcommand{\Area}{\text{Area}}
\newcommand{\vol}{\text{Vol}}




   %当然のようにやる.
\allowdisplaybreaks[4]
   %もちろん.
%\title{第1回. 多変数の連続写像 (岩井雅崇, 2020/10/06)}
%\author{岩井雅崇}
%\date{2020/10/06}
%ここまで今回の記事関係ない
\usepackage{tcolorbox}
\tcbuselibrary{breakable, skins, theorems}

\theoremstyle{definition}
\newtheorem{thm}{定理}
\newtheorem{lem}[thm]{補題}
\newtheorem{prop}[thm]{命題}
\newtheorem{cor}[thm]{系}
\newtheorem{claim}[thm]{主張}
\newtheorem{dfn}[thm]{定義}
\newtheorem{rem}[thm]{注意}
\newtheorem{exa}[thm]{例}
\newtheorem{conj}[thm]{予想}
\newtheorem{prob}[thm]{問題}
\newtheorem{rema}[thm]{補足}

\DeclareMathOperator{\Ric}{Ric}
\DeclareMathOperator{\Vol}{Vol}
 \newcommand{\pdrv}[2]{\frac{\partial #1}{\partial #2}}
 \newcommand{\drv}[2]{\frac{d #1}{d#2}}
  \newcommand{\ppdrv}[3]{\frac{\partial #1}{\partial #2 \partial #3}}


%ここから本文.
\begin{document}
\begin{center}
{\Large 第1回追加資料. 極限に関する厳密な定義 (三宅先生の本, 1.4の内容)}
\end{center}

\begin{flushright}
 岩井雅崇 2021/04/13
\end{flushright}

\section{はじめに}
この追加資料は第2回の内容を含みます.
またかなり難しい部分もあるので理解できなくても構いません.
(この内容を飛ばしてもらっても構いません.)
私はこの授業において追加資料の内容($\epsilon$-$\delta$論法等)はほぼ使いません.
後期の先生によってはこの回の内容を使う可能性もあるので, その場合にはこの資料を見ていただければ幸いです.

\subsection{数列の極限と$\epsilon$-$N$論法}

  \begin{tcolorbox}[
    colback = white,
    colframe = green!35!black,
    fonttitle = \bfseries,
    breakable = true]
    \begin{dfn}[$\epsilon$-$N$論法を用いた厳密な極限の定義]
\underline{数列が$\{a_n\}_{n=1}^{\infty}$が極限$\alpha \in \R$を持つ}とは, 
任意の正の実数$\epsilon $について, ある$N \in \N$があって, $N < n$ならば
$|a_n - \alpha| <\epsilon$となること.
このとき $$
\lim_{n \rightarrow \infty} a_n =\alpha \text{\,\,\,と書く.}$$
 \end{dfn}
 \end{tcolorbox}

 
\begin{exa}
$a_n = \frac{1}{n}$とする. 数列$\{ a_n\}$は0に収束する.

\hspace{-18pt}(証.) 
任意の$\epsilon >0$について
$N = [\frac{1}{\epsilon}] + 1$をおくと
$
\frac{1}{N} = \frac{1}{ [\frac{1}{\epsilon}] + 1 } \leqq \frac{1}{\frac{1}{\epsilon}} = \epsilon$であるため,
$$
N<n\text{\,\,\, ならば}|a_n -0| = \left|\frac{1}{n}-0\right| < \frac{1}{N} \leqq \epsilon \text{\,\,\, となる.}
$$


以上より, 任意の$\epsilon >0$について, ある$N$(具体的には$[\frac{1}{\epsilon}] + 1$)があって, $N < n$ならば$|a_n - 0| <\epsilon$となるので, 
数列$\{ a_n\}$は0に収束する.
\end{exa}

 
 
  \begin{tcolorbox}[
    colback = white,
    colframe = green!35!black,
    fonttitle = \bfseries,
    breakable = true]
    \begin{prop}
  $\lim_{n \rightarrow \infty} a_n = \alpha$, 
    $\lim_{n \rightarrow \infty} b_n = \beta$とするとき$\lim_{n \rightarrow \infty} (a_n + b_n) = \alpha + \beta$となる.
\end{prop}
 \end{tcolorbox}
 \hspace{-18pt}(証.) 
任意の$\epsilon >0$について
ある$N_1, N_2$があって
$$
N_1 < n \text{\,\,\, ならば} |a_n - \alpha | < \frac{\epsilon}{2}
$$
$$
N_2 < n \text{\,\,\, ならば} |b_n - \beta | < \frac{\epsilon}{2}
$$
 となる. 以上より$N = \max(N_1, N_2)$とおくと
 $N<n$ならば
 $$
 |(a_n + b_n) -  (\alpha + \beta)|
 \leqq |a_n - \alpha| + |b_n - \beta| <  \frac{\epsilon}{2} +  \frac{\epsilon}{2}
 = \epsilon
 $$
 である.
 以上より, 任意の$\epsilon >0$について, ある$N$ (具体的には$\max(N_1, N_2)$)があって, 
 $N < n$ならば$ |(a_n + b_n) -  (\alpha + \beta)| <\epsilon$となるので, 
数列$\{ a_n + b_n\}$は$\alpha + \beta$に収束する.

授業で紹介した収束の極限の性質の証明は上のようにやれば良い.

  \begin{tcolorbox}[
    colback = white,
    colframe = green!35!black,
    fonttitle = \bfseries,
    breakable = true]
    \begin{prop}[極限の一意性]
  $\lim_{n \rightarrow \infty} a_n = \alpha$, 
    $\lim_{n \rightarrow \infty} a_n = \beta$ならば$\alpha = \beta$である.
\end{prop}
 \end{tcolorbox}
 
  \hspace{-18pt}(証.) 
 $\alpha \neq \beta$として矛盾を示す.
% $\alpha < \beta$と仮定して良い.
 $\epsilon  = \frac{|\alpha - \beta |}{3}$とおくと, ある$N_1, N_2$があって
$$
N_1 < n \text{ならば} |a_n - \alpha | < \frac{\epsilon}{3}
\text{\,\,かつ\,\,}
N_2 < n \text{ならば} |a_n - \beta | < \frac{\epsilon}{3}
\text{ となる.}
$$
 以上より$m = \max(N_1, N_2) + 1$とおくと
 $N_1 <m$かつ$N_2 <m$より
 $$
 |\alpha - \beta|
 \leqq |a_m - \alpha| + |a_m- \beta| <  \frac{\epsilon}{3} +  \frac{\epsilon}{3}
 = \frac{2}{3} |\alpha - \beta|
 $$
 である. しかし$ |\alpha - \beta|>0$より矛盾である.



 
  \begin{tcolorbox}[
    colback = white,
    colframe = green!35!black,
    fonttitle = \bfseries,
    breakable = true]
    \begin{thm}[はさみうちの原理.]
$a_n \leqq b_n \leqq c_n$となる数列$\{ a_n \}$, $\{ b_n \}$, $\{ c_n \}$
に関して
$\lim_{n \rightarrow \infty }a_n = \lim_{n \rightarrow \infty }c_n =\alpha$
ならば
$\lim_{n \rightarrow \infty }b_n =\alpha$である.
 \end{thm}
 \end{tcolorbox}
   \hspace{-18pt}(証.) 
任意の$\epsilon >0$について
ある$N_1, N_2$があって
$$
N_1 < n \text{ならば} |a_n - \alpha | < \epsilon
\text{\,\,かつ\,\,}
N_2 < n \text{ならば} |c_n - \alpha | < \epsilon
\text{ となる.}
$$
 以上より$N = \max(N_1, N_2)$とおくと
 $N<n$ならば
 $a_n - \alpha \leqq b_n -\alpha \leqq c_n - \alpha $であるので
 $$
 |b_n -\alpha |
 \leqq \max (|a_n - \alpha| , |c_n - \alpha |) < \epsilon
 $$
 である.
 以上より, 任意の$\epsilon >0$について, ある$N$ (具体的には$\max(N_1, N_2)$)があって, 
 $N < n$ならば$ |b_n - \alpha| <\epsilon$となるので, 
数列$\{ b_n\}$は$\alpha $に収束する.
 
 
授業でちょっとだけ触れたコーシー列や実数の構成に関しても触れておきます.
 \begin{tcolorbox}[
    colback = white,
    colframe = green!35!black,
    fonttitle = \bfseries,
    breakable = true]
    \begin{dfn}[コーシー列]
数列\underline{$\{ a_n\}$がコーシー列}とは, 任意の$\epsilon >0$について, ある$N \in \N$があって, $N < m,n$ならば$|a_n - a_m| < \epsilon$となること.
 \end{dfn}
 \end{tcolorbox}
   \begin{tcolorbox}[
    colback = white,
    colframe = green!35!black,
    fonttitle = \bfseries,
    breakable = true]
    \begin{prop}[収束するならばコーシー列]
  $\lim_{n \rightarrow \infty} a_n = \alpha$ならば$\{ a_n\}$はコーシー列.
\end{prop}
 \end{tcolorbox}
    \hspace{-18pt}(証.) 
任意の$\epsilon >0$について
ある$N$があって
$$
N< n \text{\,\,\, ならば} |a_n - \alpha | < \frac{\epsilon}{2}
$$
となる. 以上より$N <n,m$ならば
 $$
 |a_n -a_m |
 \leqq |a_n - \alpha|  + |a_m - \alpha | < \epsilon
 $$
となるので, 
数列$\{ a_n\}$はコーシー列である.

\begin{exa}
逆に「コーシー列は収束するのか?」と思うが
これはどの世界で数列を考えているかによる.
有理数列$a_n$がコーシー列であっても, 数列$\{ a_n\}$が\underline{有理数には収束しない}こともあります.

例として数列$\{ a_n\}$を
$$
a_n = \text{$\sqrt{2}$の小数第$n$位まで}
$$
とおく. 
具体的には
$$
a_1 = 1.4, a_2 = 1.41, a_3=1.414, a_4 = 1.4142, \cdots
$$
である. このとき$a_n$は有理数列でありコーシー列だが
$a_n$は$\sqrt{2}$に収束するため, 
\underline{$a_n $は有理数には収束しない.}
(もちろん実数には収束してます)
\end{exa}

よって有理数の世界だけ考えても解析をするには少々不便である.(極限操作をするから.)
したがってどんなコーシー列でも収束し, 有理数を含む最小の世界があれば良いと思われる.
その思いからできたのが実数である.

\begin{tcolorbox}[
    colback = white,
    colframe = green!35!black,
    fonttitle = \bfseries,
    breakable = true]
    \begin{thm}[実数の存在]
 $\Q$を有理数の集合とする.
このとき$\Q$を含む集合$X$があって, 次を満たす.
\begin{enumerate}
\item 任意の$x \in X$に関して, ある有理数列$\{ a_n\}$があり, $\lim_{n \rightarrow \infty} a_n = x$となる.
\item $X$上の数列$\{ a_n\}$がコーシー列ならば, ある$\alpha  \in X$があり, $\lim_{n \rightarrow \infty} a_n = \alpha$となる. (コーシー列は収束する.)
\end{enumerate}

この\underline{$X$を$\R$と書き, 実数の集合}と呼ぶ.

%ここで数列$\{ a_n\}$がコーシー列とは任意の正の実数$\epsilon \in \R$について, ある$N \in \N$があって, $N < m,n$ならば$|a_n - a_m| < \epsilon$となる数列のこととする.

 \end{thm}
 \end{tcolorbox}
 \footnote{この証明は集合と位相という数学科の2年くらいで学ぶ内容です. 証明は難しいです.}


 \begin{tcolorbox}[
    colback = white,
    colframe = green!35!black,
    fonttitle = \bfseries,
    breakable = true]
    \begin{thm}[実数の連続性]
上に有界な単調増加数列$\{a_n\}$は収束する.

 \end{thm}
 \end{tcolorbox}
 
  \hspace{-18pt}(証.) 
$a_n$がコーシー列であることを示す.
$\{ a_n \}$は上に有界なので, $a_n<0$として良い.
もしコーシー列でないとすると, ある$\epsilon>0$があり, 任意の$N$について$N<n<m$となる$n,m$があって$|a_n - a_m| \geqq \epsilon $となる.

そこで新たに数列$\{b_l\}$を次のように定義する.
まず$1<n_1<m_1$となる$n_1,m_1$があって$|a_{n_1} - a_{m_1}| \geqq \epsilon $である.
よって, $b_1 = a_{n_1}, b_2 = a_{m_1}$とおく.
次に$k_2 = m_{1}+1$とおくと,
$k_2<n_2<m_2$となる$n_2,m_2$があって$|a_{n_2} - a_{m_2}| \geqq \epsilon $である.
よって, $b_3 = a_{n_2}, b_4 = a_{m_2}$とおく.
これを繰り返し行うことで帰納的に数列$\{b_l\}$を定める.

構成方法から$\{b_l\}$は単調増加で, $b_l <0$である.
さらに任意の自然数$l$について, $b_{2l} - b_{2l-1} \geqq \epsilon$かつ
$b_{2l +1} - b_{2l} \geqq 0$である.
以上より任意の自然数$l$について
$$
b_{2l} = (b_{2l} - b_{2l-1}) + (b_{2l-1} - b_{2l-2}) + \cdots + (b_2 - b_1)
+b_1 \geqq
b_1 + l\epsilon$$
である.
 $b_{2l }<0$のため, 任意の自然数$l$について
$
b_1 + l\epsilon <0
$
である.
 しかし, $\epsilon>0$であったため, これは矛盾である.
 

\section{関数の極限}

\begin{tcolorbox}[
    colback = white,
    colframe = green!35!black,
    fonttitle = \bfseries,
    breakable = true]
    \begin{dfn}[$\epsilon$-$\delta$論法を用いた厳密な極限の定義]
$f(x)$を$x=a$の周りで定義された関数とする.
\underline{$f(x)$が$x=a$で$\alpha \in \R$に収束する}とは
任意の正の実数$\epsilon$について, ある正の実数$\delta$があって, 
$|x - a|< \delta$ならば
$|f(x)- \alpha| <\epsilon$となること.
このとき $$
\lim_{x \rightarrow a} f(x) =\alpha \text{\,\,\,と書く.}$$
 \end{dfn}
 \end{tcolorbox}
 
 \begin{exa}
$f(x) = x^2$は$x=0$で0に収束する.

\hspace{-18pt}(証.) 
任意の$\epsilon >0$について
$\delta = \sqrt{\epsilon}$をおくと
$|x - 0| < \delta$ならば
$$
|f(x)-0| = |x^2| < \delta^2 = \epsilon \text{\,\,\, となる.}
$$
以上より, 任意の$\epsilon >0$について, ある$\delta$(具体的には$\sqrt{\epsilon}$)があって, $|x - 0| < \delta$ならば$|f(x)-0| <\epsilon$となるので, 
関数$f(x) = x^2$は$x=0$で0に収束する.
\end{exa}


 
  \begin{tcolorbox}[
    colback = white,
    colframe = green!35!black,
    fonttitle = \bfseries,
    breakable = true]
    \begin{prop}
  $\lim_{x \rightarrow a} f(x) = \alpha$, 
    $\lim_{x \rightarrow a} g(x)= \beta$とするとき$\lim_{x \rightarrow a}
     (f(x) + g(x)) = \alpha + \beta$となる.
\end{prop}
 \end{tcolorbox}
 \hspace{-18pt}(証.) 
任意の$\epsilon >0$について
ある$\delta_1, \delta_2 >0$があって
$$
| x - a| < \delta_1\text{ならば} |f(x)- \alpha | < \frac{\epsilon}{2}
\text{\,\,かつ\,\,}
| x - a| < \delta_2\text{ならば} |g(x)- \beta | < \frac{\epsilon}{2}
\text{\,\,となる. }
$$
以上より$\delta = \min(\delta_1, \delta_2)$とおくと, $| x - a| < \delta$ならば
 $$
 |(f(x) + g(x)) -  (\alpha + \beta)|
 \leqq |f(x) - \alpha| + |g(x) - \beta| <  \frac{\epsilon}{2} +  \frac{\epsilon}{2}
 = \epsilon
 $$
 である.
 以上より, 任意の$\epsilon >0$について, ある$\delta$ (具体的には$\min(\delta_1, \delta_2)$)があって, 
 $| x - a| < \delta$ならば$ |(f(x) + g(x)) -  (\alpha + \beta)| <\epsilon$となるので, 
$\lim_{x \rightarrow a} (f(x) + g(x)) = \alpha + \beta$となる.


授業で紹介した収束の極限の性質の証明は上のようにやれば良い.


\section{最後に}
少々書きすぎてしまったが, この内容は理解する必要はないです.
この内容が必要になることはあまりないと思います.\footnote{まあ一種の無駄知識と思っていただければ幸いです. 私はこの内容が一番面白いですが...}
 

\end{document}
