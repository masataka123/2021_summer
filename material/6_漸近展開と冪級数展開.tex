\documentclass[dvipdfmx,a4paper,11pt]{article}
\usepackage[utf8]{inputenc}
%\usepackage[dvipdfmx]{hyperref} %リンクを有効にする
\usepackage{url} %同上
\usepackage{amsmath,amssymb} %もちろん
\usepackage{amsfonts,amsthm,mathtools} %もちろん
\usepackage{braket,physics} %あると便利なやつ
\usepackage{bm} %ラプラシアンで使った
\usepackage[top=30truemm,bottom=30truemm,left=25truemm,right=25truemm]{geometry} %余白設定
\usepackage{latexsym} %ごくたまに必要になる
\renewcommand{\kanjifamilydefault}{\gtdefault}
\usepackage{otf} %宗教上の理由でmin10が嫌いなので


\usepackage[all]{xy}
\usepackage{amsthm,amsmath,amssymb,comment}
\usepackage{amsmath}    % \UTF{00E6}\UTF{0095}°\UTF{00E5}\UTF{00AD}\UTF{00A6}\UTF{00E7}\UTF{0094}¨
\usepackage{amssymb}  
\usepackage{color}
\usepackage{amscd}
\usepackage{amsthm}  
\usepackage{wrapfig}
\usepackage{comment}	
\usepackage{graphicx}
\usepackage{setspace}
\setstretch{1.2}


\newcommand{\R}{\mathbb{R}}
\newcommand{\Z}{\mathbb{Z}}
\newcommand{\Q}{\mathbb{Q}} 
\newcommand{\N}{\mathbb{N}}
\newcommand{\C}{\mathbb{C}} 
\newcommand{\Sin}{\text{Sin}^{-1}} 
\newcommand{\Cos}{\text{Cos}^{-1}} 
\newcommand{\Tan}{\text{Tan}^{-1}} 
\newcommand{\invsin}{\text{Sin}^{-1}} 
\newcommand{\invcos}{\text{Cos}^{-1}} 
\newcommand{\invtan}{\text{Tan}^{-1}} 
\newcommand{\Area}{\text{Area}}
\newcommand{\vol}{\text{Vol}}




   %当然のようにやる.
\allowdisplaybreaks[4]
   %もちろん.
%\title{第1回. 多変数の連続写像 (岩井雅崇, 2020/10/06)}
%\author{岩井雅崇}
%\date{2020/10/06}
%ここまで今回の記事関係ない
\usepackage{tcolorbox}
\tcbuselibrary{breakable, skins, theorems}

\theoremstyle{definition}
\newtheorem{thm}{定理}
\newtheorem{lem}[thm]{補題}
\newtheorem{prop}[thm]{命題}
\newtheorem{cor}[thm]{系}
\newtheorem{claim}[thm]{主張}
\newtheorem{dfn}[thm]{定義}
\newtheorem{rem}[thm]{注意}
\newtheorem{exa}[thm]{例}
\newtheorem{conj}[thm]{予想}
\newtheorem{prob}[thm]{問題}
\newtheorem{rema}[thm]{補足}

\DeclareMathOperator{\Ric}{Ric}
\DeclareMathOperator{\Vol}{Vol}
 \newcommand{\pdrv}[2]{\frac{\partial #1}{\partial #2}}
 \newcommand{\drv}[2]{\frac{d #1}{d#2}}
  \newcommand{\ppdrv}[3]{\frac{\partial #1}{\partial #2 \partial #3}}


%ここから本文.
\begin{document}
%\maketitle
\begin{center}
{\Large 第6回. 漸近展開とべき級数展開 (三宅先生の本, 2.4の内容)}
\end{center}

\begin{flushright}
 岩井雅崇 2021/05/25
\end{flushright}

\section{漸近展開とべき級数展開}

\begin{tcolorbox}[
    colback = white,
    colframe = green!35!black,
    fonttitle = \bfseries,
    breakable = true]
    \begin{thm}[有限テイラー展開]
$f(x)$が開区間$I$上の$C^n$級関数とする.
$a \in I$を固定する.
任意の$x \in I$について, ある$\theta \in (0,1)$があって
\begin{align*}
\begin{split}
f(x) &= f(a) + f'(a) (x-a) + \frac{f''(a)}{2!}(x-a)^2 + \cdots \\
&\cdots +  \frac{f^{(n-1)}(a)}{(n-1)!}(x-a)^{n-1} + \frac{f^{(n)}(a + \theta(x-a))}{n!}(x-a)^{n} \\
&=\sum_{k=0}^{n-1}\frac{f^{(k)}(a)}{k!}(x-a)^k + \frac{f^{(n)}(a + \theta(x-a))}{n!}(x-a)^{n}
\end{split}
\end{align*}
となる.
右辺を$x=a$における\underline{有限テーラー展開}と呼び, 
$R_n=\frac{f^{(n)}(a + \theta(x-a))}{n!}(x-a)^{n}$を\underline{剰余項}と呼ぶ.
特に$a=0$のとき, \underline{有限マクローリン展開}と呼ぶ.
    \end{thm}
 \end{tcolorbox}
    
\begin{tcolorbox}[
    colback = white,
    colframe = green!35!black,
    fonttitle = \bfseries,
    breakable = true]
    \begin{dfn}[ランダウの記号]
$a$を実数または$\pm \infty$とし, $f(x)$と$g(x)$を$a$の周りで定義された関数とする.
$\lim_{x \rightarrow a} \frac{f(x)}{g(x)} =0$であるとき
$$
f(x) = o(g(x))\text{\,\,\,} (x \rightarrow a) \text{\,\,と書く.}
$$
    \end{dfn}
 \end{tcolorbox}
 
 \begin{exa}
 \begin{itemize}
 \item $x^5 = o(x^3) \text{\,\,\,} (x \rightarrow 0) $
 \item $\sin x = x + o(x^2) \text{\,\,\,} (x \rightarrow 0) $
 \item 任意の正の実数$\alpha$について, $\log x = o(x^{\alpha}) \text{\,\,\,} (x \rightarrow +\infty) $であり, $x = o(e^{\alpha x}) \text{\,\,\,} (x \rightarrow +\infty)$である.
 \end{itemize}

 \end{exa}

\begin{tcolorbox}[
    colback = white,
    colframe = green!35!black,
    fonttitle = \bfseries,
    breakable = true]
    \begin{prop}[ランダウの記号の性質]
$m,n \in \N$とする.
 \begin{itemize}
 \item $x^m o(x^n) = o(x^{m+n}) \text{\,\,\,} (x \rightarrow 0) $
 \item $o(x^m) o(x^n) = o(x^{m+n}) \text{\,\,\,} (x \rightarrow 0) $
 \item $m \leqq n$ならば$o(x^m) + o(x^n) = o(x^{m}) \text{\,\,\,} (x \rightarrow 0) $
 \end{itemize}
    \end{prop}
 \end{tcolorbox}
 
\begin{tcolorbox}[
    colback = white,
    colframe = green!35!black,
    fonttitle = \bfseries,
    breakable = true]
    \begin{thm}[漸近展開]
$f(x)$を$a$を含む開区間上の$C^n$級関数ならば
\begin{align*}
\begin{split}
f(x) &= f(a) + f'(a) (x-a) + \frac{f''(a)}{2!}(x-a)^2 + \cdots +  \frac{f^{(n)}(a)}{n!}(x-a)^{n} + o((x-a)^n) \text{\,\,\,}(x \rightarrow a) \\
\end{split}
\end{align*}
となる.
特に$a=0$の場合は下のようになる.
\begin{align*}
\begin{split}
f(x) &= f(0) + f'(0) x + \frac{f''(0)}{2!}x^2 + \cdots +  \frac{f^{(n)}(0)}{n!}x^{n} + o(x^n) \text{\,\,\,}(x \rightarrow 0) \\
\end{split}
\end{align*}
    \end{thm}
 \end{tcolorbox}
 
 \begin{exa}
\begin{align*}
\begin{split}
e^x &= 1 + x+  \frac{x^2}{2!} + \frac{x^3}{3!}  + \cdots  +  \frac{ x^{n}}{n!} + o(x^{n}) \text{\,\,\,}(x \rightarrow 0) \\
\sin x &= x - \frac{x^3}{3!} + \frac{x^5}{5!} - \cdots  + 
 \frac{(-1)^{n-1} x^{2n-1}}{(2n-1)!} 
 + o(x^{2n-1}) \text{\,\,\,}(x \rightarrow 0) 
\end{split}
\end{align*}
 \end{exa}

\begin{tcolorbox}[
    colback = white,
    colframe = green!35!black,
    fonttitle = \bfseries,
    breakable = true]
    \begin{thm}[べき級数展開]
$f(x)$を$a$を含む開区間上の$C^{\infty}$級関数とする.
テイラーの定理
$$
f(x)=\sum_{k=0}^{n-1}\frac{f^{(k)}(a)}{k!}(x-a)^k + \frac{f^{(n)}(a + \theta(x-a))}{n!}(x-a)^{n}
$$
において, 剰余項$R_n(x)=\frac{f^{(n)}(a + \theta(x-a))}{n!}(x-a)^{n}$とする.
$b\in I$において$\lim_{n \rightarrow \infty}|R_n(b)| =0$となるならば,
$$
f(b)=\sum_{k=0}^{ \infty }\frac{f^{(k)}(a)}{k!}(b-a)^k  \text{\,\,\, となる.}
$$
    \end{thm}
 \end{tcolorbox}
 
 \begin{exa}
 $f(x)=e^x$とし, $a=0$かつ$b \in \R$とする. このとき剰余項は
 $$
 R_n(b) = \frac{e^{b \theta} b^n}{n!}
 $$
 である. $\lim_{n \rightarrow \infty}|R_n(b)| =0$であるので, べき級数展開ができ, 
\begin{align*}
\begin{split}
e^b &= \sum_{k=0}^{ \infty }\frac{f^{(k)}(a)}{k!}b^k = 1 + b + \frac{b^2}{2!} + \frac{b^3}{3!} + \frac{b^4}{4!} + \cdots 
\end{split}
\end{align*}
 \end{exa}
 
  \begin{exa}
 $f(x)=\sin x$とし, $a=0$かつ$b \in \R$とする. このとき剰余項は
 $$
 R_{2n}(b) = \frac{(-1)^{n}b^{2n} \sin (b \theta) }{(2n)!}
 $$
 である. $\lim_{n \rightarrow \infty}|R_n(b)| =0$であるので, べき級数展開ができ, 
\begin{align*}
\begin{split}
\sin b &= \sum_{k=0}^{ \infty }\frac{f^{(k)}(a)}{k!}b^k = 
 b - \frac{b^3}{3!} + \frac{b^5}{5!} -  \frac{b^7}{7!} +  \cdots 
\end{split}
\end{align*}
 \end{exa}
 
 \section{初等関数の漸近展開}

 初等関数の$a=0$の周りでの漸近展開の具体例を紹介する.\footnote{なんでもかんでも綺麗に漸近展開できるとは限らない. 例えば$\tan x$などの漸近展開の一般項は非常に難しい.}
\begin{align*}
\begin{split}
e^x &= 1 + x+  \frac{x^2}{2!} + \frac{x^3}{3!}  + \cdots  + 
 \frac{ x^{n}}{n!} + o(x^{n}) \text{\,\,\,}(x \rightarrow 0) \\
\sin x &= x - \frac{x^3}{3!} + \frac{x^5}{5!} - \cdots  + 
 \frac{(-1)^{n-1} x^{2n-1}}{(2n-1)!} 
 + o(x^{2n-1}) \text{\,\,\,}(x \rightarrow 0) \\
 \cos x &= 1 - \frac{x^2}{2!} + \frac{x^4}{4!} - \cdots  + 
 \frac{(-1)^{n} x^{2n}}{(2n)!} 
 + o(x^{2n}) \text{\,\,\,}(x \rightarrow 0) \\
 \log(1+x) &= x - \frac{x^2}{2} + \frac{x^3}{3}  - \cdots   
 + \frac{ (-1)^{n-1}x^{n}}{n} + o(x^{n}) \text{\,\,\,}(x \rightarrow 0) \\
  \sinh x &= x + \frac{x^3}{3!} + \frac{x^5}{5!} + \cdots  + 
 \frac{x^{2n-1}}{(2n-1)!} 
 + o(x^{2n-1}) \text{\,\,\,}(x \rightarrow 0) \\
\end{split}
\end{align*}

 
\section{演習問題}
演習問題の解答は授業の黒板にあります.
\begin{enumerate}
\item 
$$
\frac{1}{1-x} = 1 + x+  x^2 + x^3  + \cdots   
 + x^{n} + o(x^{n}) \text{\,\,\,}(x \rightarrow 0) 
$$
となることを示せ.

\end{enumerate}



 

\end{document}
