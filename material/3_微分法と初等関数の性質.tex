\documentclass[dvipdfmx,a4paper,11pt]{article}
\usepackage[utf8]{inputenc}
%\usepackage[dvipdfmx]{hyperref} %リンクを有効にする
\usepackage{url} %同上
\usepackage{amsmath,amssymb} %もちろん
\usepackage{amsfonts,amsthm,mathtools} %もちろん
\usepackage{braket,physics} %あると便利なやつ
\usepackage{bm} %ラプラシアンで使った
\usepackage[top=30truemm,bottom=30truemm,left=25truemm,right=25truemm]{geometry} %余白設定
\usepackage{latexsym} %ごくたまに必要になる
\renewcommand{\kanjifamilydefault}{\gtdefault}
\usepackage{otf} %宗教上の理由でmin10が嫌いなので


\usepackage[all]{xy}
\usepackage{amsthm,amsmath,amssymb,comment}
\usepackage{amsmath}    % \UTF{00E6}\UTF{0095}°\UTF{00E5}\UTF{00AD}\UTF{00A6}\UTF{00E7}\UTF{0094}¨
\usepackage{amssymb}  
\usepackage{color}
\usepackage{amscd}
\usepackage{amsthm}  
\usepackage{wrapfig}
\usepackage{comment}	
\usepackage{graphicx}
\usepackage{setspace}
\setstretch{1.2}


\newcommand{\R}{\mathbb{R}}
\newcommand{\Z}{\mathbb{Z}}
\newcommand{\Q}{\mathbb{Q}} 
\newcommand{\N}{\mathbb{N}}
\newcommand{\C}{\mathbb{C}} 
\newcommand{\Sin}{\text{Sin}^{-1}} 
\newcommand{\Cos}{\text{Cos}^{-1}} 
\newcommand{\Tan}{\text{Tan}^{-1}} 
\newcommand{\invsin}{\text{Sin}^{-1}} 
\newcommand{\invcos}{\text{Cos}^{-1}} 
\newcommand{\invtan}{\text{Tan}^{-1}} 
\newcommand{\Area}{\text{Area}}
\newcommand{\vol}{\text{Vol}}




   %当然のようにやる.
\allowdisplaybreaks[4]
   %もちろん.
%\title{第1回. 多変数の連続写像 (岩井雅崇, 2020/10/06)}
%\author{岩井雅崇}
%\date{2020/10/06}
%ここまで今回の記事関係ない
\usepackage{tcolorbox}
\tcbuselibrary{breakable, skins, theorems}

\theoremstyle{definition}
\newtheorem{thm}{定理}
\newtheorem{lem}[thm]{補題}
\newtheorem{prop}[thm]{命題}
\newtheorem{cor}[thm]{系}
\newtheorem{claim}[thm]{主張}
\newtheorem{dfn}[thm]{定義}
\newtheorem{rem}[thm]{注意}
\newtheorem{exa}[thm]{例}
\newtheorem{conj}[thm]{予想}
\newtheorem{prob}[thm]{問題}
\newtheorem{rema}[thm]{補足}

\DeclareMathOperator{\Ric}{Ric}
\DeclareMathOperator{\Vol}{Vol}
 \newcommand{\pdrv}[2]{\frac{\partial #1}{\partial #2}}
 \newcommand{\drv}[2]{\frac{d #1}{d#2}}
  \newcommand{\ppdrv}[3]{\frac{\partial #1}{\partial #2 \partial #3}}


%ここから本文.
\begin{document}
%\maketitle
\begin{center}
{\Large 第3回. 微分法と初等関数の性質 (三宅先生の本, 1.3と2.1の内容)}
\end{center}

\begin{flushright}
 岩井雅崇 2021/04/27
\end{flushright}

\section{微分法}
%以下この授業を通してよく使う記号や用語をまとめる.(興味がなければ飛ばして良い)

%\subsection{関数}


\begin{tcolorbox}[
    colback = white,
    colframe = green!35!black,
    fonttitle = \bfseries,
    breakable = true]
    \begin{dfn}
 $f(x)$を点$a$を含む開区間上の関数とする.
 \underline{$f(x)$が$x=a$で微分可能}とは
    $$ \lim_{x \rightarrow a} \frac{f(x) - f(a)}{x-a} \text{\,\,が存在すること.\,\,} $$
    この値を$f'(a)$と書く.
    $f'(a)$は$\drv{f}{x}|_{x=a}$や$\drv{f(a)}{x}$とも書く.
    
 \hspace{12pt}\underline{$f(x)$が$I$上で微分可能}とは, 任意の$a \in I$に関して
 $f(x)$が$x=a$で微分可能であること. このとき
  $$
\begin{array}{cccc}
f': &I& \rightarrow & \R  \\
&x& \longmapsto & f'(x)
\end{array}
$$
を\underline{$f(x)$の導関数}という. $f'(x)$は$\drv{f}{x}$とも書く.
    \end{dfn}
\end{tcolorbox}

   \begin{exa}
   みんながよく知っている関数は(だいたい)微分可能関数. つまり$x^2,\sin x, \cos x, e^x $などは微分可能な関数である.
   \end{exa}

  \begin{exa}
微分可能な関数$f(x)$について, 点$(a, f(a))$での接線の方程式は
$
y - f(a) = f'(a) (x-a) 
$
である.
   \end{exa}


 \begin{tcolorbox}[
    colback = white,
    colframe = green!35!black,
    fonttitle = \bfseries,
    breakable = true]
    \begin{thm}
$f(x)$が$x=a$で微分可能ならば$x=a$で連続である.
\end{thm}
  \end{tcolorbox}

 \begin{tcolorbox}[
    colback = white,
    colframe = green!35!black,
    fonttitle = \bfseries,
    breakable = true]
    \begin{prop}[微分の性質]
$f,g$を区間$I$上の微分可能な関数とするとき, 以下が成り立つ. ($c$は定数.)
 \begin{itemize}
 \item  $(f \pm g)' = f '\pm g'$
  \item  $(c f)' = cf'$
   \item  $(fg)' = f'g + fg'$
    \item $ \left( \frac{f}{g} \right)' = \frac{f'g - f g'}{g^2}$ 
    ($g'(x) \neq 0$なる点において.)
 \end{itemize}
 \end{prop}
   \end{tcolorbox}
   
 \begin{tcolorbox}[
    colback = white,
    colframe = green!35!black,
    fonttitle = \bfseries,
    breakable = true]
    \begin{thm}[合成関数の微分法]
$y=f(x)$が$x=a$で微分可能であり, $z=g(y)$が$y=f(a)$で微分可能であるとき, 
$z=g(f(x))$は$x=a$で微分可能であり, 
$$
\drv{z}{x}=\drv{z}{y}\drv{y}{x} 
\text{\,\,である.}
$$
より詳しく書くと, 
$$\drv{z}{x}\Bigr|_{x=a}=\drv{z}{y}\Bigr|_{y=f(a)}\drv{y}{x}\Bigr|_{x=a}
\text{\,\,である.}
$$
 \end{thm}
   \end{tcolorbox}
 
 \begin{exa}
 $z=\cos(x^2)$を普通に微分すると, $\drv{z}{x}=-2x \sin (x^2)$.
 一方$y=x^2, z=\cos y$とすると$\drv{y}{x}=2x, \drv{z}{y}=-\sin (y)$より, 
 $$
 \drv{z}{y}\drv{y}{x}  = (-\sin (x^2) )2x=-2x\sin (x^2) \text{\,\,である.}
 $$
 \end{exa}
 
  \begin{tcolorbox}[
    colback = white,
    colframe = green!35!black,
    fonttitle = \bfseries,
    breakable = true]
    \begin{thm}[合成関数の微分法]
関数$f(x)$は区間$I$で微分可能かつ単調増加であるとする.
任意の$x \in I$で$f'(x) \neq 0$であると仮定する.
このとき逆関数$f^{-1}(y)$は$f^{-1}(I)$上で微分可能であり
$$
\drv{x}{y}=\left(\drv{y}{x}\right)^{-1}=\frac{1}{\left(\drv{y}{x}\right)}
\text{\,\,である.}
$$
同じことだが, 
$$
\drv{f^{-1}}{y}=\left(\drv{f}{x}\right)^{-1}=\frac{1}{\left(\drv{f}{x}\right)}
\text{\,\,である.}
$$
 \end{thm}
   \end{tcolorbox}
   
   
 \section{初等関数の性質}
 \subsection{三角関数}
   
 \begin{tcolorbox}[
    colback = white,
    colframe = green!35!black,
    fonttitle = \bfseries,
    breakable = true]
    \begin{prop}[三角関数の微分]
    \text{}
 \begin{itemize}
 \item  $(\sin x)' = \cos x$ 
 \item  $(\cos x)' = -\sin x$
  \item  $(\tan x)' = \frac{1}{(\cos x)^2}$
 \end{itemize}
 \end{prop}
   \end{tcolorbox}

\subsection{逆三角関数}
$\sin x$は$[- \frac{\pi}{2}, \frac{\pi}{2}]$上で単調増加, 
$\cos x$は$[0, \pi]$上で単調増加, 
$\tan x$は$[- \frac{\pi}{2}, \frac{\pi}{2}]$上で単調増加
であるのでそれぞれ微分可能な逆関数が存在する.

 \begin{tcolorbox}[
    colback = white,
    colframe = green!35!black,
    fonttitle = \bfseries,
    breakable = true]
    \begin{dfn}[逆三角関数]
    \text{}
 \begin{itemize}
 \item    $$
\begin{array}{cccc}
\Sin: &[-1,1]& \rightarrow & \R  \\
&y& \longmapsto & \Sin y
\end{array}
$$
 を$\sin$の逆関数とする. これを\underline{アークサイン}と呼ぶ.
 $\Sin([-1,1])=[- \frac{\pi}{2}, \frac{\pi}{2}]$である.
 \item    $$
\begin{array}{cccc}
\Cos: &[-1,1]& \rightarrow & \R  \\
&y& \longmapsto & \Cos y
\end{array}
$$
 を$\cos$の逆関数とする. これを\underline{アークコサイン}と呼ぶ.
 $\Cos([-1,1])=[0, \pi]$である.
  \item    $$
\begin{array}{cccc}
\Tan: &\R& \rightarrow & \R  \\
&y& \longmapsto & \Tan y
\end{array}
$$
 を$\tan$の逆関数とする. これを\underline{アークタンジェント}と呼ぶ.
 $\Tan(\R)=(- \frac{\pi}{2}, \frac{\pi}{2})$である.
 \end{itemize}
 \end{dfn}
   \end{tcolorbox}
   
\begin{exa}
$\Sin(\frac{1}{2})=\frac{\pi}{6}, \Cos(\frac{1}{2})=\frac{\pi}{3},\Tan(1)=\frac{\pi}{4}$である.
\end{exa}

 \begin{tcolorbox}[
    colback = white,
    colframe = green!35!black,
    fonttitle = \bfseries,
    breakable = true]
    \begin{prop}[逆三角関数の微分]
    \text{}
 \begin{itemize}
 \item  $(\Sin y)' =  \frac{1}{\sqrt{1-y^2}}$ 
 \item  $(\Cos y)' = - \frac{1}{\sqrt{1-y^2}}$
  \item  $(\Tan y)' = \frac{1}{1 + y^2}$
 \end{itemize}
 \end{prop}
   \end{tcolorbox}
   
 \subsection{指数関数}
 
 \begin{tcolorbox}[
    colback = white,
    colframe = green!35!black,
    fonttitle = \bfseries,
    breakable = true]
    \begin{thm}[ネピアの定数]
$
\lim_{n \rightarrow \infty} \left(1 + \frac{1}{n}\right)^n
$
は収束する. この値を$e$と書きネピアの定数という.
\end{thm}
  \end{tcolorbox} 
  
 \begin{tcolorbox}[
    colback = white,
    colframe = green!35!black,
    fonttitle = \bfseries,
    breakable = true]
    \begin{dfn}[指数関数・対数関数]
    \text{}
 \begin{itemize}
\item $a>0$かつ$a \neq 1$なる実数$a$について, 関数
$$
\begin{array}{cccc}
a^x: &\R& \rightarrow & (0, + \infty)  \\
&x& \longmapsto & a^x
\end{array}
$$
を\underline{指数関数}と呼ぶ.
$a=e$のとき, $e^x$を$\exp x$ともかく.
\item $a>0$かつ$a \neq 1$なる実数$a$について, 指数関数$a^x$の逆関数
$$
\begin{array}{cccc}
\log_{a} y: &(0, + \infty) & \rightarrow & \R \\
&y& \longmapsto & \log_{a} y
\end{array}
$$
を\underline{対数関数}と呼ぶ.
$a=e$のとき, $\log y$と書く.
 \end{itemize}
 \end{dfn}
   \end{tcolorbox}
   
    \begin{tcolorbox}[
    colback = white,
    colframe = green!35!black,
    fonttitle = \bfseries,
    breakable = true]
    \begin{prop}[指数関数・対数関数の微分]
    \text{}
 \begin{itemize}
 \item  $\lim_{x \rightarrow 0} \frac{\log (1+x)}{x} =1$, $\lim_{x \rightarrow 0} \frac{e^x -1}{x} =1$.
 \item  $(a^x)' = (\log a) a^x$. 特に$(e^x)' = e^x$.
  \item  $(\log_{a} y)' = \frac{1}{(\log a) y}$. 特に$(\log y)' = \frac{1}{y}$.
 \end{itemize}
 \end{prop}
   \end{tcolorbox}

 \subsection{双曲線関数}
 
 \begin{tcolorbox}[
    colback = white,
    colframe = green!35!black,
    fonttitle = \bfseries,
    breakable = true]
    \begin{dfn}[双曲線関数]
    \text{}
 \begin{itemize}
 \item    $$
\sinh x = \frac{e^x - e^{-x}}{2}
$$
 とし, これを\underline{ハイパボリックサイン}と呼ぶ.
 \item    $$
\cosh x = \frac{e^x + e^{-x}}{2}
$$
 とし, これを\underline{ハイパボリックコサイン}と呼ぶ.
 \item    $$
\tanh x = \frac{\sinh x}{\cosh x}= \frac{e^x - e^{-x}}{e^x + e^{-x}}
$$
 とし, これを\underline{ハイパボリックタンジェント}と呼ぶ.
 \end{itemize}
 \end{dfn}
   \end{tcolorbox}
   
       \begin{tcolorbox}[
    colback = white,
    colframe = green!35!black,
    fonttitle = \bfseries,
    breakable = true]
    \begin{prop}[双曲線関数の微分]
    \text{}
 \begin{itemize}
\item $(\cosh x)^2 - (\sinh x)^2 = 1$ 
 \item  $(\sinh x)' = \cosh x$
 \item  $(\cosh x)' = \sinh x$
  \item  $(\tanh x)' = \frac{1}{(\cosh x)^2}$
 \end{itemize}
 \end{prop}
   \end{tcolorbox}


 
\section{演習問題}
演習問題の解答は授業の黒板にあります.
\begin{enumerate}
\item $\Sin(- \frac{\sqrt{3}}{2}), \Cos(- \frac{\sqrt{3}}{2}), \Tan(- \frac{\sqrt{3}}{3})$の値を求めよ.
\item $f(x) = \log(\log (x))$とする. $f'(x)$を求めよ.
\end{enumerate}



 

\end{document}
