\documentclass[dvipdfmx,a4paper,11pt]{article}
\usepackage[utf8]{inputenc}
%\usepackage[dvipdfmx]{hyperref} %リンクを有効にする
\usepackage{url} %同上
\usepackage{amsmath,amssymb} %もちろん
\usepackage{amsfonts,amsthm,mathtools} %もちろん
\usepackage{braket,physics} %あると便利なやつ
\usepackage{bm} %ラプラシアンで使った
\usepackage[top=30truemm,bottom=30truemm,left=25truemm,right=25truemm]{geometry} %余白設定
\usepackage{latexsym} %ごくたまに必要になる
\renewcommand{\kanjifamilydefault}{\gtdefault}
\usepackage{otf} %宗教上の理由でmin10が嫌いなので


\usepackage[all]{xy}
\usepackage{amsthm,amsmath,amssymb,comment}
\usepackage{amsmath}    % \UTF{00E6}\UTF{0095}°\UTF{00E5}\UTF{00AD}\UTF{00A6}\UTF{00E7}\UTF{0094}¨
\usepackage{amssymb}  
\usepackage{color}
\usepackage{amscd}
\usepackage{amsthm}  
\usepackage{wrapfig}
\usepackage{comment}	
\usepackage{graphicx}
\usepackage{setspace}
\setstretch{1.2}


\newcommand{\R}{\mathbb{R}}
\newcommand{\Z}{\mathbb{Z}}
\newcommand{\Q}{\mathbb{Q}} 
\newcommand{\N}{\mathbb{N}}
\newcommand{\C}{\mathbb{C}} 
\newcommand{\Sin}{\text{Sin}^{-1}} 
\newcommand{\Cos}{\text{Cos}^{-1}} 
\newcommand{\Tan}{\text{Tan}^{-1}} 
\newcommand{\invsin}{\text{Sin}^{-1}} 
\newcommand{\invcos}{\text{Cos}^{-1}} 
\newcommand{\invtan}{\text{Tan}^{-1}} 
\newcommand{\Area}{\text{Area}}
\newcommand{\vol}{\text{Vol}}




   %当然のようにやる.
\allowdisplaybreaks[4]
   %もちろん.
%\title{第1回. 多変数の連続写像 (岩井雅崇, 2020/10/06)}
%\author{岩井雅崇}
%\date{2020/10/06}
%ここまで今回の記事関係ない
\usepackage{tcolorbox}
\tcbuselibrary{breakable, skins, theorems}

\theoremstyle{definition}
\newtheorem{thm}{定理}
\newtheorem{lem}[thm]{補題}
\newtheorem{prop}[thm]{命題}
\newtheorem{cor}[thm]{系}
\newtheorem{claim}[thm]{主張}
\newtheorem{dfn}[thm]{定義}
\newtheorem{rem}[thm]{注意}
\newtheorem{exa}[thm]{例}
\newtheorem{conj}[thm]{予想}
\newtheorem{prob}[thm]{問題}
\newtheorem{rema}[thm]{補足}

\DeclareMathOperator{\Ric}{Ric}
\DeclareMathOperator{\Vol}{Vol}
 \newcommand{\pdrv}[2]{\frac{\partial #1}{\partial #2}}
 \newcommand{\drv}[2]{\frac{d #1}{d#2}}
  \newcommand{\ppdrv}[3]{\frac{\partial #1}{\partial #2 \partial #3}}


%ここから本文.
\begin{document}
%\maketitle
\begin{center}
{\Large 第11回. 広義積分 (三宅先生の本, 3.3の内容)}
\end{center}

\begin{flushright}
 岩井雅崇 2021/06/29
\end{flushright}



\section{広義積分}

 \begin{tcolorbox}[
    colback = white,
    colframe = green!35!black,
    fonttitle = \bfseries,
    breakable = true]
    \begin{dfn}[広義積分]
 $a$を実数とし, $b$は実数または$b=+\infty$とする. $f(x)$を$[a,b)$上の連続関数とする.
左極限$\lim_{\beta \rightarrow b-0} \int_{a}^{\beta} f(x)dx$が存在するとき, 
\underline{広義積分 $\int_{a}^{b} f(x)dx$は収束する}といい
$$
\int_{a}^{b} f(x)dx = \lim_{\beta \rightarrow b-0} \int_{a}^{\beta} f(x)dx \text{\,\,とする.}
$$
この積分を\underline{広義積分}という.
極限が存在しないときは, \underline{広義積分$\int_{a}^{b} f(x)dx$は発散する}という.
 \end{dfn}
 \end{tcolorbox}
 

 \begin{exa}
  \begin{itemize}
 \item $\int_{1}^{\infty} x^p dx$は$p<-1$のとき収束し, $p \geqq -1$のとき発散する.
 \item $\int_{0}^{1} x^p dx$は$p>-1$のとき収束し, $p \leqq -1$のとき発散する.
\end{itemize}
 \end{exa}


  \begin{tcolorbox}[
    colback = white,
    colframe = green!35!black,
    fonttitle = \bfseries,
    breakable = true]
    \begin{thm}
$f(x)$を$[a,b)$上の連続関数とする.
$[a,b)$上の連続関数$g(x)$があって, $[a , b)$上で$|f(x)| \leqq g(x)$かつ広義積分$\int_{a}^{b} g(x) dx$が収束すると仮定する.
このとき広義積分$\int_{a}^{b} f(x) dx$もまた収束する.
 \end{thm}
 \end{tcolorbox}
 
   \begin{tcolorbox}[
    colback = white,
    colframe = green!35!black,
    fonttitle = \bfseries,
    breakable = true]
    \begin{thm}
$f(x)$を$[a,b)$上の連続関数とする.
$[a,b)$上の連続関数$g(x)$があって, $[a , b)$上で$0 \leqq g(x) \leqq f(x)$かつ広義積分$\int_{a}^{b} g(x) dx$が発散すると仮定する.
このとき広義積分$\int_{a}^{b} f(x) dx$もまた発散する.
 \end{thm}
 \end{tcolorbox}
 
 \begin{exa}
広義積分$\int_{0}^{1} \frac{\sin x}{\sqrt{1-x}}dx $は収束する.
これは$[0, 1)$上で$|\frac{\sin x}{\sqrt{1-x}}| \leqq \frac{1}{\sqrt{1-x}}$かつ
広義積分$\int_{0}^1 \frac{1}{\sqrt{1-x}} dx$が収束するからである.
 \end{exa}
 
  \begin{exa}
広義積分$\int_{2}^{\infty} \frac{1}{\sqrt[3]{x(x-1)}}dx $は発散する.
これは$[2, \infty)$上で
$0 \leqq x^{-\frac{2}{3}}\leqq \frac{1}{\sqrt[3]{x(x-1)}}$
かつ
広義積分$\int_{2}^{\infty} x^{- \frac{2}{3}} dx$が発散するからである.
 \end{exa}

  \begin{exa}
実数$s>0$について, 広義積分$\int_{0}^{\infty} e^{-x}x^{s-1}dx $は収束する.

\hspace{-18pt}(証.)
$\lim_{x \rightarrow \infty} (e^{-x}x^{s-1}) x^{2} = \lim_{x \rightarrow \infty} e^{-x}x^{s+1} =0$
より, ある$c>0$があって, $[c +\infty)$上で$e^{-x} x^{s-1} \leqq x^{-2}$である.
広義積分$\int_{c}^{\infty} x^{-2} dx$は収束するため, 広義積分$\int_{c}^{\infty}e^{-x} x^{s-1}dx$も収束する.

一方$(0 , c]$上で$e^{-x}x^{s-1} \leqq x^{s-1}$であり, $s-1 > -1$から広義積分$\int_{0}^{c} x^{s-1} dx$は収束するため
広義積分$\int_{0}^{c} e^{-x}x^{s-1} dx$も収束する.

以上より広義積分$\int_{0}^{\infty} e^{-x}x^{s-1}dx  =\int_{0}^{c} e^{-x}x^{s-1}dx +\int_{c}^{\infty} e^{-x}x^{s-1}dx  $は収束する.
 \end{exa}



\section{演習問題}
演習問題の解答は授業の黒板にあります.
\begin{enumerate}
\item[] $p$を実数とし$f(x) = x^p \log x$とする.
\item $p< -1$ならば広義積分$\int_{1}^{\infty} f(x) dx$は収束することを示せ.
\item $p\geqq -1$ならば広義積分$\int_{1}^{\infty} f(x) dx$は発散することを示せ.
\end{enumerate}



 

\end{document}
