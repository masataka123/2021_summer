\documentclass[dvipdfmx,a4paper,11pt]{article}
\usepackage[utf8]{inputenc}
%\usepackage[dvipdfmx]{hyperref} %リンクを有効にする
\usepackage{url} %同上
\usepackage{amsmath,amssymb} %もちろん
\usepackage{amsfonts,amsthm,mathtools} %もちろん
\usepackage{braket,physics} %あると便利なやつ
\usepackage{bm} %ラプラシアンで使った
\usepackage[top=30truemm,bottom=30truemm,left=25truemm,right=25truemm]{geometry} %余白設定
\usepackage{latexsym} %ごくたまに必要になる
\renewcommand{\kanjifamilydefault}{\gtdefault}
\usepackage{otf} %宗教上の理由でmin10が嫌いなので


\usepackage[all]{xy}
\usepackage{amsthm,amsmath,amssymb,comment}
\usepackage{amsmath}    % \UTF{00E6}\UTF{0095}°\UTF{00E5}\UTF{00AD}\UTF{00A6}\UTF{00E7}\UTF{0094}¨
\usepackage{amssymb}  
\usepackage{color}
\usepackage{amscd}
\usepackage{amsthm}  
\usepackage{wrapfig}
\usepackage{comment}	
\usepackage{graphicx}
\usepackage{setspace}
\setstretch{1.2}


\newcommand{\R}{\mathbb{R}}
\newcommand{\Z}{\mathbb{Z}}
\newcommand{\Q}{\mathbb{Q}} 
\newcommand{\N}{\mathbb{N}}
\newcommand{\C}{\mathbb{C}} 
\newcommand{\Sin}{\text{Sin}^{-1}} 
\newcommand{\Cos}{\text{Cos}^{-1}} 
\newcommand{\Tan}{\text{Tan}^{-1}} 
\newcommand{\invsin}{\text{Sin}^{-1}} 
\newcommand{\invcos}{\text{Cos}^{-1}} 
\newcommand{\invtan}{\text{Tan}^{-1}} 
\newcommand{\Area}{\text{Area}}
\newcommand{\vol}{\text{Vol}}




   %当然のようにやる.
\allowdisplaybreaks[4]
   %もちろん.
%\title{第1回. 多変数の連続写像 (岩井雅崇, 2020/10/06)}
%\author{岩井雅崇}
%\date{2020/10/06}
%ここまで今回の記事関係ない
\usepackage{tcolorbox}
\tcbuselibrary{breakable, skins, theorems}

\theoremstyle{definition}
\newtheorem{thm}{定理}
\newtheorem{lem}[thm]{補題}
\newtheorem{prop}[thm]{命題}
\newtheorem{cor}[thm]{系}
\newtheorem{claim}[thm]{主張}
\newtheorem{dfn}[thm]{定義}
\newtheorem{rem}[thm]{注意}
\newtheorem{exa}[thm]{例}
\newtheorem{conj}[thm]{予想}
\newtheorem{prob}[thm]{問題}
\newtheorem{rema}[thm]{補足}

\DeclareMathOperator{\Ric}{Ric}
\DeclareMathOperator{\Vol}{Vol}
 \newcommand{\pdrv}[2]{\frac{\partial #1}{\partial #2}}
 \newcommand{\drv}[2]{\frac{d #1}{d#2}}
  \newcommand{\ppdrv}[3]{\frac{\partial #1}{\partial #2 \partial #3}}


%ここから本文.
\begin{document}
%\maketitle
\begin{center}
{\Large 第5回. 高次導関数とテイラーの定理 (三宅先生の本, 2.3と2.4の内容)}
\end{center}

\begin{flushright}
 岩井雅崇 2021/05/18
\end{flushright}

\section{高次導関数}

\begin{tcolorbox}[
    colback = white,
    colframe = green!35!black,
    fonttitle = \bfseries,
    breakable = true]
    \begin{dfn}[高次導関数の定義]
$f(x)$を区間$I$上の微分可能な関数とする.
$f'(x)$が$I$上で微分可能であるとき, \underline{$f$は2回微分可能}であるといい, 
$$
f''(x) = (f'(x)  )'
$$
としてこれを\underline{2次の導関数}と呼ぶ.
$f''(x)$は$f^{(2)}(x)$とも書く.

\hspace{12pt}同様に$f^{(n-1)}(x)$が微分可能であるとき, \underline{$f$は$n$回微分可能}であるといい, 
\underline{$n$次導関数} $f^{(n)}(x)$を$(f^{(n-1)}(x))'$として定める.
 $f^{(n)}(x)$は$\drv{^n f}{x^n}$とも書く.
    \end{dfn}
\end{tcolorbox}

\begin{exa}
\begin{itemize}
\item $f(x) = e^x$とすると, $  f^{(n)}(x) = e^x$である.
\item $f(x) = \sin x$とすると, 
   $$
  f^{(n)}(x)= \begin{cases}
(-1)^m \sin x& (n = 2m) \\
   (-1)^m \cos x& (n = 2m+1)
  \end{cases}
  \text{\,\,である.}
  $$
\end{itemize}
\end{exa}

\begin{tcolorbox}[
    colback = white,
    colframe = green!35!black,
    fonttitle = \bfseries,
    breakable = true]
    \begin{dfn}[$C^n$級関数]
$f(x)$を区間$I$上の関数とする.
\begin{itemize}
\item $f(x)$が$n$回微分可能であり, $f^{(n)}(x)$が連続であるとき, 
\underline{$f$は$C^n$級関数}であるという.
\item 任意の$n \in \N$について$f$が$C^n$級であるとき, 
\underline{$f$を$C^{\infty}$級関数}であるという.
\end{itemize}

    \end{dfn}
\end{tcolorbox}

\begin{exa}
みんながよく知っている関数は(だいたい)$C^{\infty}$級関数. 
つまり$x^2,\sin x, \cos x, e^x $などは$C^{\infty}$級関数である.
\end{exa}


\section{テイラーの定理とその応用}

\begin{tcolorbox}[
    colback = white,
    colframe = green!35!black,
    fonttitle = \bfseries,
    breakable = true]
    \begin{thm}[テイラーの定理 1]
$f(x)$が開区間$I$上の$C^2$級関数とする.
$a<b$なる$a,b \in I$について
$$
f(b) = f(a) + f'(a) (b-a) + \frac{f''(c)}{2}(b-a)^2
$$
となる$c \in (a,b)$が存在する.
    \end{thm}
\end{tcolorbox}

\begin{exa}
$f(x) = e^x$とし$a=0$かつ$b$を正の実数とする.
このときある$c \in (0,b)$があって
$$
e^b = f(0) + f'(0) b + \frac{f''(c)}{2}b^2
= 1 + b + \frac{e^c}{2} b^2
$$
となる. $e^c \geqq 1$であるため, 
$$
e^b \geqq  1 + b + \frac{1}{2} b^2 \text{\,\,となる.}
$$
\end{exa}

\begin{tcolorbox}[
    colback = white,
    colframe = green!35!black,
    fonttitle = \bfseries,
    breakable = true]
    \begin{thm}[極値判定法]
$f(x)$が点$a$の周りで定義された$C^2$級関数とする.
\begin{itemize}
\item $f'(a) = 0$かつ$f''(a)>0$なら$f(x)$は$x=a$で極小.
\item $f'(a) = 0$かつ$f''(a) < 0$なら$f(x)$は$x=a$で極大.
\end{itemize}
    \end{thm}
\end{tcolorbox}

\begin{tcolorbox}[
    colback = white,
    colframe = green!35!black,
    fonttitle = \bfseries,
    breakable = true]
    \begin{thm}[テイラーの定理 2]
$f(x)$が開区間$I$上の$C^n$級関数とする.
$a<b$なる$a,b \in I$について
$$
f(b) = f(a) + f'(a) (b-a) + \frac{f''(a)}{2!}(b-a)^2 + \cdots 
+  \frac{f^{(n-1)}(a)}{(n-1)!}(b-a)^{n-1} + \frac{f^{(n)}(c)}{n!}(b-a)^{n}
$$
となる$c \in (a,b)$が存在する.
    \end{thm}
\end{tcolorbox}

\begin{exa}
$f(x) = e^x$とし$a=0$かつ$b$を正の実数とする.
このときある$c \in (0,b)$があって
\begin{align*}
\begin{split}
e^b &= f(0) + f'(0) b + \frac{f''(0)}{2!}b^2 +  \cdots +\frac{f^{(n-1)}(0)}{(n-1)!}b^{n-1} + \frac{f^{(n)}(c)}{n!}b^{n} \\
&= 1 + b + \frac{1}{2!} b^2 + \frac{1}{3!} b^3 + \cdots 
+ \frac{1}{(n-1)!}b^{n-1}  + \frac{e^c}{n!}b^{n} 
\end{split}
\end{align*}
となる. $e^c \geqq 1$であるため, 
$$
e^b \geqq  1 + b + \frac{1}{2!} b^2 + \frac{1}{3!} b^3 + \cdots 
+ \frac{1}{(n-1)!}b^{n-1}  + \frac{1}{n!}b^{n} 
\text{\,\,となる.}
$$
\end{exa}

\begin{tcolorbox}[
    colback = white,
    colframe = green!35!black,
    fonttitle = \bfseries,
    breakable = true]
    \begin{thm}[有限テイラー展開]
$f(x)$が開区間$I$上の$C^n$級関数とする.
$a \in I$を固定する.
任意の$x \in I$について, ある$\theta \in (0,1)$があって
\begin{align*}
\begin{split}
f(x) &= f(a) + f'(a) (x-a) + \frac{f''(a)}{2!}(x-a)^2 + \cdots \\
&\cdots +  \frac{f^{(n-1)}(a)}{(n-1)!}(x-a)^{n-1} + \frac{f^{(n)}(a + \theta(x-a))}{n!}(x-a)^{n} \\
&=\sum_{k=0}^{n-1}\frac{f^{(k)}(a)}{k!}(x-a)^k + \frac{f^{(n)}(a + \theta(x-a))}{n!}(x-a)^{n}
\end{split}
\end{align*}
となる.
右辺を$x=a$における\underline{有限テーラー展開}と呼び, 
$R_n=\frac{f^{(n)}(a + \theta(x-a))}{n!}(x-a)^{n}$を\underline{剰余項}と呼ぶ.
特に$a=0$のとき, \underline{有限マクローリン展開}と呼ぶ.
    \end{thm}
 \end{tcolorbox}
    


\section{演習問題}
演習問題の解答は授業の黒板にあります.
\begin{enumerate}
\item 
任意の$x \in \R$についてある$\theta \in (0,1)$があって
$$
\sin x = 1 - \frac{x^3}{3!} + \frac{x^5}{5!} - \cdots  + 
 \frac{(-1)^{n-1} x^{2n-1}}{(2n-1)!} 
 + \frac{ (-1)^n x^{2n}\sin (\theta x) }{2n!}
$$
となることを示せ.

\end{enumerate}



 

\end{document}
